\chapter{Introduction}

Change point detection is an important practical problem across many domains and applications.
And it is tightly connected to the Concept drift (CD) problem in machine learning area.
Change point detection is mainly for time series data (one- or multi-dimensional).
While CD is more general problem to detect changes in high-dimensional data streams feeded into ML predictive models.

% 1. Motivation
Time series data are widespread as it is a natural form to store information about any evolving phenomenon.
Even if the data set used to train a predictive model is assumed to be temporary static, underlying assumption is that in the future the data distribution will change and the model will be re-trained. Snapshots used to trained the model over time are indexed by time moment when they were relevant and form a high dimensional data stream containing data snapshots indexed by time.

Historically, on-line change detection in time series data is an old practical problem with the roots in the problem of statistical quality control~\cite{basseville1993detection}, ~\cite{NISTbook}. Walter A. Shewhart invented control charts in 1924 while working on the problem of statistical quality control to improve reliability of telephone transmission systems.

Quality control itself is even more old and dates back to Middle Ages when manufacturing using machines began.
In the modern time quality control is transferred to the area of Machine Learning with the aim to control and maintain quality of predictive models.

In the era of Machine Learning the problem of change detection in time series data is transferred to the area of predictive models as Concept drift problem.

In batch, or, offline training settings, models are trained using a fixed training data set and then used to make predictions on incoming data. 
However, streaming, or, on-line settings, are more prevalent in real world applications. 
In both cases a concept drift can happen and the model performance can drop. 
In order to detect concept drifts, change detectors are used to monitor changes in models performance metrics on-line~\cite{gama2004learning}. 
Once change is alarmed the model can be re-trained using the most relevant data between last change and current time moment.
In this scenario model will be trained using only most recent relevant data batch and will have a better performance than sliding window or growing window updated model.
We propose to use recurrency information to skip false alarms and possibly detect changes with smaller delays.

(SDM)
The task of online change point detection in sensor data streams is often complicated due to presence of noise that can be mistaken for real changes and therefore affecting performance of change detectors.
Increasing volumes of data are being generated by sensors when monitoring industrial processes, traffic, or infrastructure.
Such streaming data typically needs to be analyzed continuously as it arrives, using only information observed so far.
Predictive models, once learned on historical data, may become obsolete due to changes in physical process generating data, as well as external changes in the environment.
In machine learning and data mining changes in the underlying data distribution over time are referred as concept drift~\cite{Widmer96}.
A popular approach for handling concept drift is to monitor data or model performance for changes and to adapt model using most recent data collected after the last detected change~\cite{GamaACMCS2014}.
Even without automated model adaptation, online change detection is practically relevant in many domains, such as medicine, energy production, industrial processes monitoring~\cite{Nikiforov}, for signalling human experts that something has changed in the process.
Change points are defined as moments in time, when the statistical properties of a data stream change significantly according to predefined criteria.
The process of online change detection is often challenging due to presence of noise that, when operating online, may be mistaken for real changes.
A good change detector is expected to detect changes within an acceptable time lag, and be robust to noise by not raising false alarms.
A balance between these two requirements is determined by the parameter settings for the detector.
Typically, the optimal settings are learned during an offline training phase, and then applied on newly arriving data.
Our idea is to adjust the sensitivity of the detector online during operation, in relation to the current probability of change reoccurrence.
While many change detection methods have been developed~\cite{Nikiforov,Polunchenko2011} for offline and online settings, they typically assume that changes occur at random in time, and are independent from each other.
In practice, however, in many industrial applications changes occur with some regularity (e.g. seasonality).
Our approach is to capture this information from data, and utilize it for improving the accuracy of online change detection.
%
Consider the following example.
A continuously operating power production plant needs to be regularly refueled.
Change detection software is deployed for detecting refueling events from sensor data, and providing this information to the operation and control system.
If, for instance, historical data suggests that refueling usually happens between 7pm and 10pm every day, this information can be learned from data, and the sensitivity of the detector can be increased within this time range expecting to improve detection speed, and lowered otherwise expecting to reduce the rate of false alarms.

Online change point detection is an important task for machine learning in changing environments.
Presence of noise that can be mistaken for real changes makes it difficult to develop an effective approach that would have a low false alarm rate and being able to detect all the changes with a minimal delay.
Online change detection is practically relevant in many domains, such as medicine, energy production, industrial processes monitoring~\cite{Nikiforov}.
In machine learning and data mining research areas change detection is often studied in the context of problem of concept drift happening due to changes in the underlying data distribution over time~\cite{Widmer96}. A popular approach for handling concept drift is to monitor data or model performance for changes and to adapt model using most recent data collected after the last detected change~\cite{GamaACMCS2014}.

In the real world, the data tends to change over time.
In fact, it is almost never static.
Machine learning systems need to have mechanisms for monitoring changes in data distributions, for continuous diagnostics of their models performance, and for model adaptations that can handle concept drifts, i.e.\ sudden or gradual changes in underlying data distributions.
The so-called informed methods for handling concept drift are based on online change detection mechanisms. Most of the existing mechanisms assume that changes are not predictable.
However, there are various application settings in which concept drifts are expected to reoccur with a pattern.
Intuitively, contextual information about past and future change points locations may help to perform online change detection with shorter delays and with lower false alarm rates.
We show how this performance gain can be achieved for Cusum detector as one of the most popular detection methods based on sequential analysis, in presence of prediction intervals capturing change points locations.

Machine learning under concept drift has been an actively studied research area. \cite{gama2014survey} and \cite{zliobaite2016overview} provide comprehensive overviews of methods and studied application scenarios. As the data distributions change over time, machine learning models need to adapt to these changes in a timely manner. The so-called informed adaptation methods rely on on-line change detectors that trigger an adaptation mechanism. 

Most of the existing change detection methods assume that changes are independent from each other and occur at random in time.



% 2. Research goals / questions
Our research goal is to develop a mathematical model and its software implementation for improving change detection performance in case if CD or change event is predictable with the aim to i) reduce FA rate and ii) decrease the detection delay. 

% 3. Contributions 

The core is articles 1), 2) and 3)

% All articles and which we include
In the paper 1) we introduced Pccf and integrated it with the naive threshold based detector.
In the paper 2) we address uncertainty in Pccf parameters estimation and integrate Pccf with the Bayesian detector.
In 3) we improve Pccf definition and calculation procedure and integrate it with the Cusum detector, address detection delay decrease and particular statistical properties of the Cusum output statistic.

\begin{enumerate}[leftmargin=0.2cm]

  \item \textbf{Alexandr Maslov}, Mykola Pechenizkiy, Indr\.e~\v{Z}liobait\.e, Tommi K\"{a}rkk\"{a}inen \textbf{``Modelling Recurrent Events for Improving Online Change Detection''} (SDM 2016, SIAM International Conference on Data Mining)

  \item \textbf{Alexandr Maslov}, Mykola Pechenizkiy, Yulong Pei, Indr\.e~\v{Z}liobait\.e, Alexander Shklyaev, Tommi K\"{a}rkk\"{a}inen, Jaakko Hollm{\'e}n \textbf{``BLPA: Bayesian Learn-Predict-Adjust Method for Online  Detection of Recurrent Changepoints''} (IJCNN 2017, The 2017 International Joint Conference on Neural Networks, Anchorage, Alaska, USA, May 14-19)

  \item \textbf{Alexandr Maslov}, Mykola Pechenizkiy, Tommi K\"{a}rkk\"{a}inen \textbf{``Improving CUSUM performance in presence of prediction intervals''} (Submitted)

\end{enumerate}

\begin{enumerate}[leftmargin=0.2cm]
  \item \textbf{Alexandr Maslov}, Hoang Thanh Lam, Mykola Pechenizkiy, Eric Bouillet, Tommi K\"{a}rkk\"{a}inen \textbf{``DOBRO: A Prediction Error Correcting Robot Under Drifts''} (SAC 2016, The 31st ACM/SIGAPP Symposium on Applied Computing)

  \item Tommi K\"{a}rkk\"{a}inen, \textbf{Alexandr Maslov}, Pekka Wartiainen
\textbf{``Region of interest detection using MLP''} (ESANN 2014, European Symposium on Artificial Neural Networks: Bruges, Belgium)

  \item Ayoze Marrero, Juan Mendez, \textbf{Alexandr Maslov}, Mykola Pechenizkiy \textbf{``ACLAC: An adaptive closed-loop anesthesia control''} (CBMS 2013: Porto, Portugal)

  \item Hindra Kurniawan, \textbf{Alexandr Maslov}, Mykola Pechenizkiy \textbf{``Stress Detection from Speech and Galvanic Skin Response Signals''} (CBMS 2013, The 26th IEEE International Symposium on Computer-Based Medical Systems: Porto, Portugal)

  \item \textbf{Alexandr Maslov}, Mykola Pechenizkiy, Tommi K\"{a}rkk\"{a}inen, Matti T\"{a}htinen
%\href{http://dl.acm.org/citation.cfm?id=2350185}{}
\textbf{``Quantile Index for Gradual and Abrupt Change Detection from CFB Boiler Sensor Data in Online Settings''}
(Proceedings of the Sixth International Workshop on Knowledge Discovery from Sensor Data (ACM SIGKDD), Beijing, 2012)

  \item Denis Kotkov, \textbf{Alexandr Maslov}, Mats Neovius. 2021. \textbf{``Revisiting the Tag Relevance Prediction Problem''}. In Proceedings of the 44th International ACM SIGIR Conference on Research and Development in Information Retrieval, (SIGIR 2021). Association for Computing Machinery, New York, NY, USA, 1768–1772. %DOI:\href{https://doi.org/10.1145/3404835.3463019}
\end{enumerate}

% 4. Thesis outline



In Chapter 2 (Background) we review relevant information regarding Concept drift phenomenon, concept drift adaptation techniques, change detection problem, recurrency.

In Chapter 3 we discuss research questions and methodology.
In the Chapter 3 we formulate research questions and methodological approach for conducted research. 

In Chapter 4 we describe the developed mathematical model and implementation for Predictive Confidence Change Function (Pccf).

Chapter 5 describes how Pccf can be integrated with various change detectors (Cusum, Bayesian, Naive). 

Chapter 6 highlights Pccf application for predictive model adaptation for concept drift. 

Chapter 7 describes conclusions and future work.
