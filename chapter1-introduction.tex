\chapter{Introduction}
% 1. Motivation

In batch, or, offline training settings, models are trained using a fixed training data set and then used to make predictions on incoming data. 
However, streaming, or, on-line settings, are more prevalent in real world applications. 
In both cases a concept drift can happen and the model performance can drop. 
In order to detect concept drifts, change detectors are used to monitor changes in models performance metrics on-line~\cite{gama2004learning}. 
Once change is alarmed the model can be re-trained using the most relevant data between last change and current time moment.
In this scenario model will be trained using only most recent relevant data batch and will have a better performance than sliding window or growing window updated model.
We propose to use recurrency information to skip false alarms and possibly detect changes with smaller delays.
%Questions.
%If there is outlier and we make a prediction, then what is the correct prediction?
%If there is outlier and change is alarmed then it is considered as a change and situation is the same as depicted on Figure..
%If we change is not alarmed by having outlier, ...

% 2. Research goals

% 3. Thesis outline

In Chapter 2 (Background) we review relevant information regarding Concept drift phenomenon, concept drift adaptation techniques, change detection problem, recurrency.
% sections
There are various types of change detectors. 
Section X describes concept drift phenomenon.
Sections X,Y,Z describe Adwin, Cusum, Bayesian, etc.detectors.
Section X describes recurrency property.
Section X describes Pccf and its applications.
Section XXX describes how recurrency helps to reduce FA rate and decrease detection delay for CFB signal.

In Chapter 3 we discuss research questions and methodology.

In Chapters X,Y,Z, we describe results obtained in published papers relevant for concept drift.
