
\documentclass[12 pt]{article}
\renewcommand{\baselinestretch}{1}
\usepackage{geometry}
\usepackage{graphics}
\geometry{verbose,letterpaper,tmargin=0.5 cm,bmargin=0.8 cm,lmargin=2.5cm,rmargin=2.5cm,headsep=1cm}
\setlength{\parskip}{\smallskipamount}
\setlength{\parindent}{10 pt}
\usepackage{amssymb}
\usepackage{amsmath}
\usepackage{textcomp}
\usepackage{setspace}
\usepackage{indentfirst}
\usepackage{hyperref}

\usepackage{algorithm}% http://ctan.org/pkg/algorithms
\usepackage{algpseudocode}% http://ctan.org/pkg/algorithmicx
\usepackage{listings}
\usepackage{hyperref}
\usepackage{enumitem}
\usepackage{array}
\usepackage{tikz}
\usepackage{graphics}
\usepackage{standalone}
\usepackage{amsthm}
\usepackage{breakcites}
\begin{document}
\section{References overview}
\begin{itemize}
  \item Change detection:~\cite{basseville1993detection}
  \item Sequential change detection problem is a well studied problem, see for example in~\cite{tartakovsky2014sequential},~\cite{plasse2021streaming}.

  \item Optimality of the change detection procedure was investigated in~\cite{Page1954},~\cite{Shiryaev2010,Shiryaev1961,Shiryaev1963}.
  Asymptotic and nonasymptotic optimality of cumulative sum algorithms was provedin~\cite{lorden1971procedures},~\cite{moustakides1986optimal},~\cite{moustakides2004optimality},~\cite{ritov1990decision}. In~\cite{Shiryaev1963,shiryaev2007optimal} the change point is modelled as a random variable with a known geometric distribution~\cite{veeravalli2014quickest} and optimal algorithm minimizing the average detection delay given constraint on the probability of false alarm is proposed. In our work we minimize the detection delay given a constraint on the maximum delay imposed by the prediction interval width. In~\cite{lorden1971procedures} asymptotic optimality of Cusum~\cite{Page1954} is proved according to the minimax criterion for delay with the mean time between false alarms going to infinity.

  \item Concept drift:
\end{itemize}

\section{Related work (general)}
Building blocks are based on the papers~\cite{mackay2007},~\cite{Chapados2014},~\cite{Rasmussen2010} ,~\cite{MacKay_Inference_Book}.

HMM~\cite{HMMtutor}

In~\cite{Box_Jenkins_Arima} the state of the art ARIMA method is considered.

In~\cite{mackay2007} Bayesian approach.


According to~\cite{GamaAsurveyOnConceptDrift} 
\begin{definition}
   The concept is
   \begin{equation}
   \text{\textit{Concept}} = P(X,Y)
   \end{equation}    
\end{definition}
and according to~\cite{WebbConceptDrift} the concept drift is
\begin{definition}
   if the concept at a particular time $t$ is $P_t(X)$ then Concept drift occurs between times $t$ and $u$ when the distributions change
   \label{def:concept_drift}    
\end{definition}



  \bibliographystyle{unsrt}
  \bibliography{references}
\end{document}
