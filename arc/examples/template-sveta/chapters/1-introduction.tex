\chapter{Introduction}
%\label{chapter:introduction}

Nowadays, {\em mathematical models} are widely used to describe processes in 
different branches of natural sciences, medicine, engineering, and economics. 
Evolutionary problems, in particular, are fundamental components in simulations of 
real-life processes such as heat conduction and thermal radiation models in 
thermodynamics, global climate prediction, forecasting and understanding the 
weather, and estimation of forest growth, among others. Later examples basically 
testify the fact
that questions arising in mathematical modeling originate from and are 
highly motivated 
by the phenomena surrounding us.

%----------------------------------------------------------------------------------%
Most of the models mentioned above are governed by time-dependent {\em partial 
differential equations} (PDEs) or systems of PDEs, which in combination with initial 
(IC) and boundary conditions (BCs) produce so-called {\em initial-boundary value 
problems} (I-BVPs). 
%----------------------------------------------------------------------------------%
The current study is focused on evolutionary problems of {\em parabolic type}, the 
systematic mathematical analysis of which is presented in monographs 
\cite{Ladyzhenskaya1985, Ladyzhenskayaetall1967, Wloka1987, Zeidler1990A,Zeidler1990B}.
The numerical analysis and study of the practical application are exposed in works 
\cite{Thomee2006,Lang2001} and partially in classical books on finite element method 
(FEM) on PDEs and saddle problems (see, e.g., \cite{Braess2001,GrossmannRoosStynes2007,
Johnson2009,Glowinski1984,Glowinskietall1976}). 
The multiharmonic analysis of a distributed parabolic and optimal control problem 
in a time-periodic BVPs setting has been studied in
\cite{KollmannKolmbauerLangerWolfmayrZulehner2013,LangerWolfmayr2013}.

%----------------------------------------------------------------------------------%
Let $Q_T := \Omega \times ]0, T[$ denote the space-time cylinder, where 
$\Omega \subset \Rd$, \linebreak $d \in \{1, 2, 3\}$, is a bounded 
domain with Lipschitz boundary $\partial \Omega$, and $]0, T[$ is a given time 
interval, $0 < T < +\infty$. The cylindrical surface is denoted by $S_T$, i.e., 
$S_T := \partial \Omega \times [0, T]$. 
A general form of a {\em linear parabolic I-BVP problem} reads as follows:
%
\begin{alignat}{2}
\partial_t u + \mathcal{L} u & = f \;\qquad {\rm in} \quad Q_T , 
\label{eq:equation}\\[-2pt]
u & = u_D \;\;\quad {\rm on} \quad S_T, 
%\label{eq:Dirichlet-boundary-condition}
\\[-2pt]
u(x, 0) & = u_0 \qquad {\rm on} \quad \Omega.
\label{eq:initial-condition}
\end{alignat}
%
%----------------------------------------------------------------------------------%
Here, depending on the application, $u$ might describe the temperature alteration 
in heat conduction or the 
concentration of certain substance in chemical diffusion. The given data includes 
the source term $f$, Dirichlet BC $u_D$ (Neumann or Robin can be considered instead), 
and IC $u_0$. The elliptic operator $\mathcal{L}$ has the general form
%
$$\mathcal{L} u := 
	-\dvrg (A(x, t) \nabla u(x, t)) + b(x) \cdot \nabla u(x, t) + c(x)\, u(x, t),
	\quad (x, t) \in Q_T,
$$
%
where $A$ is the material characteristics matrix, and 
$b$ and $c$ stand for convection and reaction, 
respectively. If any of the latter forms depend on $u$ (or $\nabla u$), we arrive at 
a nonlinear problem. 
%----------------------------------------------------------------------------------%

For $b \equiv 0$, $c \equiv 0$, and $A = \nu I$, we obtain a heat equation which governs 
diffusion processes. For instance, in heat conduction applications the parameter 
$\nu = \tfrac{k}{c_p \varrho}$ stands for thermal 
diffusivity \cite{Fourier1952, CarslowJaeger1948, Widder1975, Cannon1984}, 
in electromagnetics it illustrates resistivity $\nu = \tfrac{1}{\sigma}$, which is 
inversely related to the electric conductivity of material $\sigma$.
%, a material-specific 
%quantity depending on the thermal $k$, specific heat capacity $c_p$ , and mass 
%density of the material $\varrho$ 
%reserve 
%of electric conductivity constant $\sigma$, which is very low in the air 
%(an excellent insulator), and very high in metals (in plasma is treated as infinite).
Moreover, the heat equation is used in propagation of action potential in nerve 
cells, phenomena arising in finance, e.g., the Black--Scholes 
\cite{BlackScholes2012} or Ornstein-Uhlenbeck processes, probability, and description of random walks 
\cite{Pearson1905}. The nonlinear analogs of the heat equation have also been 
used in image processing and modeling of porous media \cite{Vazquez2007}.
%Some models 
%of nonlinear heat conduction (which are also parabolic equations) have solutions 
%with finite heat transmission speed, e.g., porous medium equation and the other 
%related models have solutions with finite wave propagation speed \cite{Vazquez2007}. 

%----------------------------------------------------------------------------------%
%The heat equation is, technically, in violation of special relativity, because its 
%solutions involve instantaneous propagation of a disturbance. The part of the 
%disturbance outside the forward light cone can usually be safely neglected, but if 
%it is necessary to develop a reasonable speed for the transmission of heat, 
%a hyperbolic partial differential equation should be considered instead. 

%----------------------------------------------------------------------------------%
The subject of our interest, i.e., evolutionary systems of PDEs, in majority of 
cases can only be solved in the generalized sense by one of two discretization 
techniques described below. In the first, the so-called {\em incremental time-stepping 
method}, the time is discretized by ordinary differentiation (OD) and the obtained 
reduced problem (in space coordinates) is approximated by FEM 
(\cite{Courant1943, Hrennikoff1941, 
Zlamal1968, Ciarlet1978,Johnson2009}) or the finite difference method (FDM 
\cite{Lax1948, RouleauOsterle1955, MortonMayers1994,GrossmannRoosStynes2007,
Crank1975}) on successive time sub-intervals (the detailed study of such an 
approach
can be found in the monographs 
\cite{Thomee2006, Braess2001, Johnson2009}). In the second 
method the time is considered as an additional spatial 
variable \cite{Hackbusch1984,Womble1990,VandewallePiessens1992,HortonVandewalle1995}.
It is usually referred to as the {\em space-time discretization} technique. 
%application 
%of numerical method, e.g., finite element method (FEM) 
%\cite{Courant1943, Hrennikoff1941, Zlamal1968, Ciarlet1978,, Johnson2009}, 
%finite volume method (FVM) 
%\cite{Toro1999,EymardGalloutHerbin2000, LeVeque2002}, 
%finite difference method (FDM) 
%\cite{Lax1948, RouleauOsterle1955,MortonMayers1994,GrossmannRoosStynes2007,Crank1975},
%the boundary element method (BEM) \cite{BanerjeeButterfield1981, Brebbia1978}.
Regardless of the method used, the obtained approximation contains an 
{\em error}. Therefore, it is of high importance to construct a proper numerical tool
to analyze the obtained results and to provide reliable information 
on the {\em approximation error} encompassed in it 
in order to avoid the risk of 
drawing the wrong conclusion form obtained numerical information.

%----------------------------------------------------------------------------------%
There exist two approaches for evaluating the approximation error. The 
\emph{a priori} approach is used for the qualitative verification of 
theoretical properties of the numerical method, e.g., rate of convergence and 
asymptotic behavior of the approximation with respect to mesh size parameters 
(see, 
e.g., \cite{BrennerScott1994,Ciarlet1978,StrangFix1973} 
and references cited therein). However, the high regularity requirements, which must 
be satisfied in order to apply estimates from the latter group, are quite unrealistic. 

%----------------------------------------------------------------------------------%
In the second, the so-called {\em a posteriori} approach, the error is measured after 
computation of the approximation. Unlike in a priori error analysis, the alternative 
estimates exploit only the given data, e.g., domain characteristics, source 
function together with IC and BC, and the approximation itself. The upper bound of 
the gap between the approximate and 
exact solution measured in terms of relevant energy norm is 
called an {\em error estimate} or {\em majorant}. The quantity replicating the 
distribution of the true error over the domain is called an \emph{error indicator}. 
%----------------------------------------------------------------------------------%
There are three principal ways classifying existing error indicators.  
The first, the so-called {\em residual method}, is based on the estimation of the 
residual 
functional introduced in \cite{BabushkaRheinboldt1978, BabushkaRheinboldtSIAM1978} 
and various modifications of them covered in a wealth of publications 
\cite{ErikssonJohnson1988, JohnsonHansbo1992, 
AinsworthOden1992, AinsworthOden1993, Verfurth1996, DorflerRumpf1998, 
Carstensen1999, CarstensenVerfurth1999,AinsworthOden2000,CarstensenFunken2000,
BabushkaStrouboulis2001, BabushkaWhitemanStrouboulis2011}).
%----------------------------------------------------------------------------------%
The second approach is based on the approximation of latter functional or so-called 
{\em post-processing}, e.g., gradient averaging 
\cite{ZienkiewiczZhu1987, ZienkiewiczZhu1988} and expanded in various works 
\cite{AinsworthOden1992,BabushkaRodriguez1993,Verfurth1996, Zienkiewiczetall1998,
Wang2000,BabushkaStrouboulis2001,BartelsCarstensen2002, WangYe2002, 
Heimsundetall2002, ZhangNaga2005}). Its mathematical justification relies 
on the {\em superconvergence} phenomenon 
\cite{OganesjanRuhovec1969, Zlamal1977} and actively studied in 
\cite{KrizekNeittaanmaki1984, 
KrizekNeittaanmaki1987,Krizeketall1998, Krizeketall1998,Wahlbin1995}. Other 
techniques from the second group are based on partial 
equilibration \cite{LadevezeLeguillon1983, AinsworthOden2000, Braess2001}, global 
averaging \cite{CarstensenFunken2000, BartelsCarstensen2002,Heimsundetall2002}, and 
solution of local sub-problems \cite{Ainsworth1998, AinsworthOden2000, 
AinsworthRankin2010}.
%----------------------------------------------------------------------------------%
Finally, the third method is dependent on the solution of the auxiliary problem, e.g., 
{\em hierarchically} based error indicators \cite{Deuflhardetall1989,
Agouzal2001,Duranetall1991,DorflerNochetto2002} and {\em goal-oriented} error 
\cite{BeckerRannacher1996,SteinOhnimus1997,PerairePatera1998,
Houstonetall2000,Rannacher2000,OdenPrudhomme2001,Steinetall2007, Meidneretall2009,
RannacherVexler2010,BesierRannacher2012}. The concept of a posteriori error estimation jointly with mesh-adaptive methods, which are focused on the optimization of 
computing resources, have become a well-established approach in the numerical 
analysis of PDEs. 

%, and highly overestimate
%the error. Moreover, it is very costly from the computational point of view due to 
%the fact that it is based on the local interpolation constants of the FE in the mesh.

The guaranteed error bounds for evolutionary models considered in this thesis are based 
on two different mathematical approaches. One of these follows from the theory of 
contraction mappings and the Banach fixed point theorem. The other approach pursues
the theory of functional a posteriori estimates.

%----------------------------------------------------------------------------------%
The first part of the study, in particular, is dedicated to the investigation of 
numerical treatment of the Cauchy problem with non-linearity  (see, e.g., 
\cite{CoddingtonLevinson1972, Haireretall1993, Teschl2012}), 
which can be obtained from 
\eqref{eq:equation}--\eqref{eq:initial-condition} by assuming that $\Omega$ 
coincides with $\Rd$. The so-called \PL method suggests 
one possible way to treat nonlinear ordinary differentiation 
equations (ODEs). 
It belongs to a class of iteration method and can 
be found in \cite{Liouville1838, Peano1888, Bendixson1893, Lindelof1894, 
Picard1890}. A similar idea is used for PDEs in \cite{Picard1890} and analyzed 
thoroughly in \cite[Vol.II]{Picard18911996}. 
%In \PL method transforms the 
%differential problem to the form of an integral equation, and once the contractivity 
%of integral operator is proved, the Banach theorem provide the convergence of 
%iterative procedure to the exact solution. 
The combination of the \PL method with Ostrowski a posteriori estimates provides a
fully guaranteed Adaptive \PL (APL) algorithm for solving ODEs. 
Moreover, the algorithm takes into account information about
discretization errors related to the numerical integration and interpolation.
The results obtained during the investigation of the APL method 
confirmed that it can be applied for the treatment of nonlinear 
evolutionary models, which belong to my main research topics for the future. 
Nonlinear PDEs exhibit multiple properties which do not 
appear in linear theory but are often related to important features of the real
world phenomena. In this work, we concentrate only on the linear 
models. The application of Ostrowski 
estimates is also extended to classical iteration schemes, the obtained results 
are exposed in \cite[Section 6.7]{Malietall2014}.

%----------------------------------------------------------------------------------%
Generally, the \PL method can be used not only for ODEs but also for time-dependent 
algebraic and functional equations (see, e.g., 
\cite{NevanlinnaPI1989, NevanlinnaPII1989}, where it is shown that the speed of
convergence is independent of the step sizes). Numerical methods based on \PL 
iterations for dynamical processes (the so-called waveform relaxation in the context 
of electrical networks) are discussed in \cite{Erolaetall1995}. 
A posteriori estimates and nodal superconvergence for time
stepping methods are studied in \cite{Akrivisetall2011,MakridakisNochetto2006} for
linear and nonlinear problems.

%----------------------------------------------------------------------------------%
The second and main part of this work is devoted to the \emph{functional type} a 
posteriori error estimates and indicators initially introduced by Repin in 
\cite{Repin1997, RepinPowerFunc1997, Repin1999, Repin2000} and thoroughly studied 
for various classes of problems (see, e.g., \cite{NeittaanmakiRepin2004, 
RepinDeGruyter2008, Malietall2014} and references therein). Unlike the above-listed 
error indicators, functional type error estimates are guaranteed, they do not contain 
mesh-dependent local interpolation constants (contrary to residual estimates), and 
they are valid for any function from the class of conforming approximations (not 
restricted by the Galerkin orthogonality assumption). The detailed comparison of the
above-described approaches can be found in monograph by Mali, 
Neittaanm{\"a}ki, and Repin \cite{Malietall2014}.

%----------------------------------------------------------------------------------%
Our main goal is to develop a fully reliable tool to quantitatively control the error 
in approximate solutions of evolutionary problems. The numerical treatment of this 
class of I-BVPs produces approximations, which alongside with the progress of 
simulations accumulate the error. This error may eventually `blow up' if it is 
not controlled. Therefore, the appropriate error estimates are crucial for 
monitoring its possible dramatic growth. Once the error in 
the approximation has been controlled reliably, it is possible to detect areas with 
excessively high local errors and calculate an essentially more accurate 
approximation.

%----------------------------------------------------------------------------------%
In the framework of the a posteriori error estimates studied in this work, we highlight 
the paper \cite{Repin2002}, where a method of deriving functional error 
estimates for parabolic I-BVPs is suggested. The first attempt on their numerical analysis 
is presented in \cite{GaevskayaRepin2005}. In \cite{RepinTomar2010}, the authors 
study the extension of error estimates for evolutionary convection-diffusion 
problems with 
possible discontinuity of approximations in time. A posteriori error analysis of 
parabolic time-periodic BVPs in connection with their multiharmonic FE 
discretization is presented in \cite{LangerRepinWolfmayr2014}. The residual 
estimates are also extended to evolutionary PDEs in  
\cite{Verfurth2003, BangerthRannacher2003, Vexler2008, Meidneretall2011, 
BesierRannacher2012, RichterSpringerVexler2013} and the reference cited therein.
Lastly, $hp$-Galerkin time-stepping for the same 
class of problems is addressed in \cite{Johnson1988,SchotzauSchwab2000,
SchotzauWihler2010} and references cited therein. 



%----------------------------------------------------------------------------------%
In order to make functional estimates applicable to a wider class of problems with 
$\Omega$ of complicated geometry, the domain decomposition (DD) technique in 
combination with local Poincar\'{e} inequalities is discussed in 
\cite[Section 3.5.3]{RepinDeGruyter2008}
for elliptic PDEs. The current work extends the latter 
estimates to time-dependent PDEs and suggests a method to omit both Friedrichs' and 
trace global constants, which are included into the basic form of the majorant. 
%----------------------------------------------------------------------------------% 

Suggested in \cite{RefMatculevichNeitaanmakiRepin2015} and
\cite{RefMatculevichRepinPoincare2014} method applies Poincar\'{e}-type inequalities
that in addition to quantitative analysis of PDEs is also used in various problems 
of numerical analysis, e.g., discontinuous Galerkin, mortar and DD methods. 
%----------------------------------------------------------------------------------% 
The exact values of respective constants (or sharp and guaranteed 
bounds of them) are interesting from both analytical and computational points of 
view. 
Results related to constants in extension and projection type estimates related to 
FE approximations can be found in, e.g., \cite{Mikhlin1986,Ciarlet1978}. Constants 
in the trace inequalities associated with polygonal domains are discussed in 
\cite{CarstensenSauter2004}; in FETI, FETI-DP DD methods the application of 
constants is highlighted in \cite{Klawonnatall2008,Dohrmann2008} and 
\cite{ToselliWidlund2005}. Functional inequalities and respective constants play an important role in analysis of problems described in terms of vector-valued functions 
(see, e.g., \cite{Fuchs2011, Pauly2015}). 
In \cite{KikuchiLiu2007,XuefengOishi2013}, the analysis of error constants for 
piecewise constant and linear interpolations over triangular finite elements can be 
found. And finally, \cite{CarstensenGedicke2014} introduces fully computable 
two-sided bounds on the eigenvalues of the Laplace operator based on the 
approximation of the corresponding 
eigenfunction in the nonconforming Crouzeix-Raviart FE space.


%----------------------------------------------------------------------------------%
The last part of this study is dedicated to sharp bounds of the constants in 
classical Poincar\'{e} and 
Poincar\'{e}-type inequalities for arbitrary non-degenerate triangles and tetrahedrons, which 
are typical objects in various discretization methods. These computable estimates 
are based on the mapping of the reference simplices to arbitrary one using the 
exact values of the respective constants derived in 
\cite{Pinsky1980, HoshikawaUrakawa2010,KikuchiLiu2007} for some triangles and 
\cite{NazarovRepin2014} for parallelepipeds, rectangles, and right triangles. 
Knowledge about the sharp upper bounds for the above-mentioned constants is particularly 
useful for quantitative analysis of problems generated by differential equations 
and implementation of the functional error majorants applied for the problems with decomposed domain.

Below, we sketch the structure of the thesis. 
%----------------------------------------------------------------------------------%
Chapter \ref{chapter:mathematical-background} is dedicated to the overview of the  
mathematical framework, including definitions and theorems in the field of functional 
analysis as well as results on solvability parabolic I-BVPs, which provide 
fundamental results required in the subsequent chapters.
%----------------------------------------------------------------------------------%
Chapter \ref{chapter:estimates} is focused on the main results achieved in this study, 
i.e., a fully guaranteed APL method for ODEs, functional a posteriori error estimates 
for the distance to the exact solution of parabolic I-BVPs, and sharp bounds of the 
constants in classical Poincar\'{e} and Poincar\'{e}-type inequalities for functions 
with zero mean traces on the faces of arbitrary simplexes in $\Rtwo$ and $\Rthree$. 
%----------------------------------------------------------------------------------%
In Chapter \ref{chapter:conclusion}, we draw some conclusions and give an outlook 
on future work in connection to efficient and fully guaranteed solvers for 
nonlinear evolutionary problems. The results presented in the included papers, 
or in other publications, will be highlighted accordingly. The connections between 
the topics are presented in Figure \ref{fig:structure}.
%----------------------------------------------------------------------------------%
\vskip 10pt
\noindent
{\bf Author's contribution to the included articles}
\vskip 10pt

\noindent
\cite{RefMatculevichNeittaanmakiRepin2013}: The estimates studied in this paper were 
discussed originally in monograph of Neittaanm{\"a}ki and Repin 
\cite[Section 3.1]{NeittaanmakiRepin2004}. The goal of this 
work is to implement the adaptive iterative \PL method and combine it with Ostrowski estimates. 
The computations of the numerical part are carried out in \verb+MATLAB+ \cite{Matlab}
by the author. 
Application of 
Ostrowski estimates to classical iteration schemes is also presented in 
\cite[Section 6.7.6]{Malietall2014} together with a guaranteed APL method.

%----------------------------------------------------------------------------------%
\vskip+1em
\noindent
\cite{RefMatculevichRepin2014}: This article studies functional type a 
posteriori error estimates for evolutionary reaction-diffusion I-BVP with a reaction 
function, which drastically \linebreak 
changes its values on different parts of the domain. 
The method suggested for derivation of the majorant combines ideas presented in 
original work of Repin \cite{Repin2002} on bounds of the distance to the 
exact solution of heat equation and join paper of Repin and Sauter 
\cite{RepinSauter2006}, which is concerned with state reaction-diffusion BVP. 
The minorant of the error in the 
approximate solution for the evolutionary class of problems derived in the paper is 
the original result. Its efficiency is confirmed by numerical tests. 
All experiments presented in the paper 
are implemented by the author in \verb+MATLAB+. 

%----------------------------------------------------------------------------------%
\vskip+1em
\noindent
\cite{RefMatculevichNeitaanmakiRepin2015}: 
The focus of this paper is on error estimates for an approximate solution of 
the evolutionary reaction-diffusion problem in case of decomposed domains. The 
method suggested in the paper is based on the idea originally introduced for the 
elliptic problems in \cite{RepinDeGruyter2008} and \cite{RepinSauter2006}. The main 
goal of the work is to overcome the complications arising with the calculation of 
Friedrichs' constant included in the majorant presented in 
\cite{RefMatculevichRepin2014} once 
it is applied to problems with a domain of a complicated shape. By exploiting the idea
of DD and classical Poincar\'{e} inequalities \cite{Poincare1890, Poincare1894}, 
we exclude global constants from the majorant. The proofs and technicalities 
in the paper are the work of the author.
%----------------------------------------------------------------------------------%
\vskip+1em
\noindent
\cite{RefMatculevichRepinPoincare2014}: This work is another generalization of the 
error estimates presented in \cite{RefMatculevichRepin2014} to the problems 
formulated on complicated domains with nontrivial mixed Dirichlet--Robin BC. 
Again, by using the method of domain decomposition and application of local 
Poincar\'{e}-type inequalities 
for functions with zero mean trace, we omit global trace and Friedrichs' constants 
included into the basic majorant. Besides that, we demonstrate the equivalence of 
errors measured in primal and combined norms to advanced and basic forms of majorants, 
respectively.

%----------------------------------------------------------------------------------%
\vskip+1em
\noindent
\cite{RefArxivMatculevichRepin2015}: The technique suggested in 
\cite{RefMatculevichRepinPoincare2014} and \cite{RefMatculevichNeitaanmakiRepin2015} 
is based on local Poincar\'{e} and Poincar\'{e}-type 
inequalities for functions with zero mean trace on the whole boundary or measurable 
part of it. We suggest explicit relations (based on exact constants from 
\cite{Pinsky1980,HoshikawaUrakawa2010,KikuchiLiu2007,NazarovRepin2014}) that 
serve as sharp and easily computable (independent of any discretization parameters) 
bounds of the respective constants. Moreover, we compare obtained bounds of the 
constants in the classical Poincar\'{e} inequalities with known analytical estimates
and investigate, numerically, the behavior of minimizers of Rayleigh quotients, corresponding the constants. The numerical experiments in the 
paper are carried out by the author, using both \verb+MATLAB+ 
and \verb+The FEniCS Project+ \cite{Fenicsproject,LoggMardalWells2012}.

%----------------------------------------------------------------------------------%
% tikz picture
%----------------------------------------------------------------------------------%
\vskip 30pt
\begin{figure}[h]
\centering
\input{pics/structure}
\vspace{20pt}
\caption{Structure of the thesis.}
\label{fig:structure}
\end{figure}

