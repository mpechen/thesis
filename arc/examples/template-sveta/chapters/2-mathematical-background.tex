\chapter{Mathematical background}
\label{chapter:mathematical-background}

%----------------------------------------------------------------------------------%
In this chapter, we concisely introduce the notation, mathematical framework, and 
fundamental results that form the basis for the further investigations and 
findings presented in this thesis. For detailed expositions, we refer the reader
to monographs \cite{NeittaanmakiRepin2004,RepinDeGruyter2008,Malietall2014}.

%----------------------------------------------------------------------------------%
\section{Function spaces and inequalities}
\label{sec:function-space-inequalities}

%----------------------------------------------------------------------------------%
The sections below present the definitions and main results for Sobolev spaces, 
which are used for the treatment of elliptic BVPs and parabolic I-BVPs. Although, 
some results are quite well-known, we discuss them to keep the work as self-content
as possible. In addition, for more detailed 
and fundamental presentations of the results highlighted below, we refer the reader 
to \cite{Evans2010, Zeidler1990A, Zeidler1990B, Wloka1987}.

\hiddensubsection{Spaces of integrable functions}
%\label{ssec:space-integrable-func} 

\noindent
Let $\Omega \subset \Rd$ , $d = \{1, 2, 3\}$, be a bounded domain 
with Lipschitz 
boundary $\partial \Omega$, where $\overline{\Omega}$ is the closure of $\Omega$,
and 
$\Gamma$ be a part of $\partial \Omega$ such that $\meas_{d - 1} \Gamma > 0$ 
(or in particular case may coincide with it). We note that throughout the
thesis discussions will be restricted to real spaces.
%----------------------------------------------------------------------------------%
Let $\{X, \|\cdot \|_{X}\}$ denote a {\em Banach space}, i.e., a vector space $X$ 
equipped with a norm $\|\cdot\|_X$, such that $X$ is complete with respect to it.
Let $\{V, \|\cdot\|_V\}$ denote a {\em Hilbert space}, where the norm is induced 
by the inner product 
$(\cdot, \cdot)_V: V \times V \rightarrow \Real$, i.e., 
$\|\cdot\|_V := (\cdot, \cdot)^{\rfrac{1}{2}}_V$. The space $V^*$ denotes the dual 
to $V$, consists of linear continuous functionals on $V$, and is equipped with norm 
%
%\begin{equation*}
$\|f\|_{V^*} := \Sup_{v \in V, \, v \neq 0} \tfrac{f(v)}{\|v\|_V}.
$
%\label{eq:dual-norm}
%\end{equation*}
%
The so-called duality product \linebreak
$\langle \cdot, \cdot \rangle_{V^* \times V}: V^* \times V \rightarrow \Real$ is 
defined as 
%----------------------------------------------------------------------------------%
%
\begin{equation}
\langle f, v \rangle_{V^* \times V} \, := \,f(v), \quad \forall v \in V.
\label{eq:duality-product}
\end{equation}
%
%----------------------------------------------------------------------------------%
The totality of all measurable in the Lebesgue sense functions $u$ with finite norm
%
\begin{equation*}
\| u\|_{\L{p}} := \Big( \Int_{\Omega} |u(x)|^p \dx \Big)^{\rfrac{1}{p}}.
%\label{eq:Lp-norm}
\end{equation*}
%
forms a separable Banach space and is denoted by 
$\L{p}(\Omega)$, $p \in [1, +\infty[$.
%
%--------------------------------------------------------------------------------%
For spaces of essentially bounded functions with $p = \infty$, the 
norm is defined as \linebreak
%
\begin{equation*}
\| u\|_{\L{\infty}} := {\rm ess} \Sup_{x \in \Omega} |u(x)|. 
\label{eq:Linfty-norm}
\end{equation*}
%
%--------------------------------------------------------------------------------%
% write that this space is the most restrictive one due to the embedding theorem
Further, we are mainly interested in the Hilbert space of square-integrable 
functions 
$\L{2}(\Omega)$ eqquiped with the norm 
$\| \cdot \|_{\L{2}(\Omega)} := (\cdot, \cdot)^{\rfrac{1}{2}}_{\L{2}(\Omega)}$
induced by 
%
\begin{equation*}
(u, v)_{\L{2}(\Omega)} = (u, v):= \Int_{\Omega} u \, v \dx, 
\quad \forall u, v \in \L{2}(\Omega).
%\label{eq:L2-inner-product}
\end{equation*}
%
%(??)
For the purpose of shortening the notation, in the cases of discussing 
$\L{2}$-measures on $\Omega$, the $\L{2}$-norm is denoted $\| \cdot \|_{\Omega}$. 

\hiddensubsection{Differentiability classes}
%

%--------------------------------------------------------------------------------%
\noindent
Let $\alpha = (\alpha_1, \ldots ,\alpha_d)$, $\alpha_i \in \Nd \cup {0}$, 
$i = 1, \ldots, d$, be a multi-index; then \linebreak
%
%\begin{equation*}
$	D^\alpha u 
	:= \tfrac{\partial^{|\alpha|}}{\partial x^{\alpha}} u 
	= \tfrac{\partial^{\alpha_1}}{\partial x^{\alpha_1}} \ldots 
		\tfrac{\partial^{\alpha_d}}{\partial x^{\alpha_d}} u,
$	%\label{eq:differential-operator}
%\end{equation*}
%
where $x^\alpha$ is the monomial ${x_1}^{\alpha_1} \ldots {x_d}^{\alpha_d}$ with 
degree $|\alpha| = \Sum_{i = 1}^{d} \alpha_i$. 
%
%--------------------------------------------------------------------------------%
Functions in $C^{l}(\Omega)$ possess continuous and 
bounded derivatives $D^\alpha$ up to order $l$. The space 
$C^{l}(\overline{\Omega})$ is 
equipped with the norm 
%
\begin{equation*}
	\| u \|_{C^{l}(\overline{\Omega})} := \Max_{0 \leq |\alpha| \leq l} 
	\Sup_{x \in {\overline{\Omega}}} |D^\alpha u(x)|.
\end{equation*}
%
%--------------------------------------------------------------------------------%
The norm for continuous functions ($l = 0$) is defined by 
$\| \cdot \|_{C(\overline{\Omega})}$.
The space $C^{\infty}(\Omega)$ consists of infinitely differentiable (smooth) 
functions, and elements of $C^{\infty}_{0}(\Omega) \subset C^{\infty}(\Omega)$ 
have compact support in 
$\Omega$. Smooth functions vanishing on $\Gamma$ are denoted by
%
\begin{equation}
	C^{\infty}_{0, \Gamma}(\Omega) 
	:= \Big\{ \varphi \in C^{\infty} (\Omega) \, | \, 
				{\rm dist} ({\rm supp} \varphi, \Gamma ) > 0 \Big\}.
	\label{eq:homogenous-G}
\end{equation}

\hiddensubsection{Sobolev spaces}

\noindent
The $\alpha^{th}$ weak (or generalized) derivative of $u \in \L{2}(\Omega)$ is 
denoted by \linebreak
$w = D^\alpha u \in \L{2}(\Omega)$ such that
%
\begin{equation*}
	\Int_\Omega w \, v \dx = (-1)^{|\alpha|} \Int_\Omega u \, D^\alpha v \dx, 
	\quad \forall v \in C^{\infty}_{0}(\Omega).
	%\label{eq:weak-derivative}
\end{equation*}
%
%--------------------------------------------------------------------------------%
The separable space of Banach type 
$\W{l, p}{} (\Omega)$, $p \in [1, +\infty[$ and $l \in \Nd$,
is called the {\em Sobolev space} 
%
%\begin{equation*}
$\W{l, p}{} (\Omega) 
	:= \Big\{ u \in \L{p}(\Omega) \, | \,  D^\alpha u \in \L{p}(\Omega), 
						|\alpha| \leq l \Big\}$
	%\label{eq:Wkp-space}
%\end{equation*}
%
and equipped with the norm
%
\begin{equation}
	\| u\|_{\W{l, p}{}} 
	:= \Big( \Sum_{|\alpha| \leq l } \| D^\alpha u \|^p_{\L{p}} \Big)^{\rfrac{1}{p}}. 
	\label{eq:Wlp-norm}
\end{equation}
%
If the boundary of $\Omega$ is smooth enough, 
latter space coincide with a clouser of $C^{l}(\overline{\Omega})$ 
under the norm \eqref{eq:Wlp-norm}, i.e., 
$\mathcal{W} : = \overline{C^{l}(\overline{\Omega})}^{\| \cdot \|_{\W{l, p}{}}}$  
(in general, $\mathcal{W} \subset \W{l, p}{}$).

%--------------------------------------------------------------------------------%
The Hilbert spaces with $p = 2$ are traditionally denoted as 
$\H{l}(\Omega) = \W{l,\,2}{}(\Omega)$. Later in the thesis, we use the spaces
%
\begin{alignat*}{2}
	\H{1} (\Omega) & 
	:= \Big\{ u \in \L{2}(\Omega) \, | \,  \nabla u \in \L{2}(\Omega, \Rd) \Big\}, 
	\quad {\rm and}\\
	\H{}(\dvrg, \Omega) & 
	:= \Big\{ u \in \L{2}(\Omega, \Rd) \, | \,  \dvrg u \in \L{2}(\Omega) \Big\}, 
	%\label{eq:H-1-H-div}
\end{alignat*}
%
with the corresponding norms $\| \cdot \|_{\H{1} (\Omega)}$ and 
$\| \cdot \|_{\H{} (\dvrg, \Omega)}$ induced by 
%
%\begin{alignat*}{2}
	%(u, v)_{\H{1}} & := (u, v) + (\nabla u, \nabla v) \quad {\rm and}\\
	%(u, v)_{\H{}(\dvrg)} & := (u, v) + (\dvrg u, \dvrg v), 
	%%\quad \forall u, v \in \L{2}(\Omega), 
	%%\label{eq:H-1-H-div}
%\end{alignat*}
%
\begin{equation*}
	(u, v)_{\H{1}} := (u, v) + (\nabla u, \nabla v) \quad {\rm and} \quad
	(u, v)_{\H{}(\dvrg)} := (u, v) + (\dvrg u, \dvrg v), 
\end{equation*}
%
respectively. 
Spaces with homogenous boundary conditions on 
$\Gamma \subset \partial \Omega$ are defined as closures of 
\eqref{eq:homogenous-G}:
%
%\begin{alignat*}{2}
\begin{equation*}
\HD{1}{0, \Gamma} (\Omega) := 
\overline{C^{\infty}_{0, \Gamma} (\Omega)}^{\H{1}(\Omega)} \quad {\rm and } \quad
\HD{}{0, \Gamma} (\dvrg, \Omega) := 
\overline{ C^{\infty}_{0, \Gamma} (\Omega)}^{\H{}(\dvrg, \Omega)}.
%\label{eq:H-1-H-div}
%\end{alignat*}
\end{equation*}
%
%--------------------------------------------------------------------------------%
If $\Gamma = \partial \Omega$, then 
$\HD{1}{0, \partial \Omega} (\Omega) = \HD{1}{0} (\Omega)$. 
%--------------------------------------------------------------------------------%
Lastly, let $\gamma_{\Gamma} u \in C(\Gamma)$ 
denote the restriction of $u \in C(\overline{\Omega})$ to $\Gamma$, i.e., 
$\gamma_{\Gamma} u (x) := u(x)$, $\forall x \in \Gamma$. The latter one
is called {\em trace operator}
$\gamma_{\Gamma} : \H{s} (\Omega) \rightarrow \H{s-\rfrac{1}{2}} (\Gamma)$, 
$s \in (\tfrac{1}{2}, \tfrac{3}{2})$.
%

\hiddensubsection{\bf Inequalities}
%\label{ssec:inequalities}

\noindent
We list several algebraic and functional inequalities frequently used in the 
thesis. For $a, b \in \Real$ and any positive $\beta$, we have general Young 
inequality
%
\begin{equation}
a b \leq \tfrac{1}{p} (\beta a)^p + \tfrac{1}{q} (\tfrac{b}{\beta})^q, \quad
\tfrac{1}{p} + \tfrac{1}{q} = 1.
\label{eq:young-inequality}
\end{equation}
%
Next, for any functional $\F$ and its convex conjugate $\F^*$, the Fenchel inequality 
holds 
%
\begin{equation}
\langle v^*, v \rangle_{V^* \times V } \leq \F^*(v^*) + \F(v), 
\quad \forall v^* \in V^*, \quad \forall v \in V.
\label{eq:young-fenchel-inequality}
\end{equation}
%
When last two are used in combination, they are referred as Young-Fenchel 
inequality.
%--------------------------------------------------------------------------------%

\noindent
The H\"{o}lder inequality for integrable functions reads as
%
\begin{equation}
\Int_{\Omega} u \,v \dx \leq \| u \|_{\L{p}} \| v\|_{\L{q}}, \quad
\forall u \in \L{p}(\Omega), \; \forall v \in \L{q}(\Omega), \quad
\tfrac{1}{p} + \tfrac{1}{q} = 1.
\label{eq:holder-inequality}
\end{equation}
%
For $p = q = 2$, it is referred as the Cauchy-Bunyakowski-Schwarz inequality. 

%--------------------------------------------------------------------------------%
We recall the main inequalities from the embedding theory. First,  
Friedrichs' inequality \cite{Friedrichs1937} has the form
%
\begin{equation*}
	\| u \|_{\Omega} \leq \CFriedrichs \| \nabla u \|_{\Omega},
	\quad \forall u \in \HD{1}{0}(\Omega).
	\label{eq:friedrichs-inequality}
\end{equation*}
%
The Poincar\'{e} inequality \cite{Poincare1890,Poincare1894} reads as 
%
\begin{equation}
	\| u \|_{\Omega} 
	\leq C_{\mathrm{P} \Omega} \, \| \nabla u \|_{\Omega},
	\quad \forall u \in \tildeH{1}(\Omega),
	\label{eq:classical-poincare-constant}
\end{equation}
%
%--------------------------------------------------------------------------------%
where $\tildeH{1}(\Omega) 
:= \big\{ u \in \H{1}(\Omega)\,  \big | 
          \,{  \{  u \} }_{\Omega} = 0 \;\big\}$, where
$\,{  \{  u \} }_{\Omega} := \tfrac{1}{|\Omega|} \Int_{\Omega} w \dx$. The 					
above-introduced constants $0 < \CFriedrichs := \tfrac{1}{\sqrt{\lambda^D_1}} < +\infty$
and $0 < C_{\mathrm{P}} := \tfrac{1}{\sqrt{\lambda^N_2}} < +\infty$, where 
$\lambda^D_1$ is 
the first eigenvalue of the Dirichlet-Laplacian and $\lambda^N_2$ is the second 
eigenvalue
of the 
Neumann-Laplacian. Due to inequality $0 < \lambda^N_{n+1} < \lambda^D_{n}$ for 
all $n \in \Nd$ (see \cite{Filonov2004}), the relation 
$\CFriedrichs < C_{\mathrm{P} \Omega}$ holds.
%
According to \cite{Mikhlin1986}, 
for simple bounded domain in $\Rd$ encompassed inside a 
rectangle with edges of length $l_i$, $i = 1, \ldots, d$, we have the estimate
$\CFriedrichs \leq \tfrac{1}{\pi} 
\Big(\Sum_{i = 1}^{d} l^{\minus 2}_i \Big)^{-\rfrac{1}{2}}$.
%
The Poincar\'{e} constant $C_{{\rm P}\Omega}$ can be estimated as 
$C_{{\rm P}\Omega} \leq \tfrac{\diam \Omega}{\pi}$ for convex domain 
$\Omega$ (see \cite{PayneWeinberger1960}). For simplexes in $\Rtwo$, 
this estimate was improved in \cite{LaugesenSiudeja2010}, where it was shown 
that $C_{{\rm P}\Omega} \leq \tfrac{ \diam \Omega}{j_{1, 1}}$ 
for all nondegenerated triangles, and 
%
\begin{equation*}
    C_{{\rm P}\Omega} \leq \CPLS := \diam \Omega \cdot \,
    \begin{cases}
    \tfrac{1}{j_{1, 1}} &  \alpha \in (0, \tfrac{\pi}{3}],\\
    \min \Big\{ \tfrac{1}{j_{1, 1}}, \tfrac{1}{j_{0, 1}}
                            \big(2 (\pi - \alpha) \tan(\alpha / 2)\big)^{-\rfrac{1}{2}} \Big\} &
                                                    \alpha \in (\tfrac{\pi}{3}, \tfrac{\pi}{2}], \\
    \tfrac{1}{j_{0, 1}} \big(2 (\pi - \alpha) \tan(\alpha / 2)\big)^{-\rfrac{1}{2}} &
    \alpha \in (\tfrac{\pi}{2}, \pi]\\
    \end{cases}
    %\label{eq:improved-estimates}
\end{equation*}
%
for isosceles one. %
Here, $j_{0, 1} \approx 2.4048$ and $j_{1, 1} \approx 3.8317$ are the smallest positive
roots of the Bessel functions $J_0$ and $J_1$, respectively. 

Exact value of constant in \eqref{eq:classical-poincare-constant} 
on equilateral triangle with unit side 
is derived in \cite{Pinsky1980}, i.e., $\CPT = \tfrac{3}{4 \pi}$. 
Constants for the right isosceles triangles with legs $\tfrac{\sqrt{2}}{2}$ and 
$1$ are $\CPT = \tfrac{1}{\sqrt{2}\pi}$ and $\CPT = \tfrac{1}{\pi}$, respectively. 
The latter one can be found from \cite{HoshikawaUrakawa2010} and \cite{KikuchiLiu2007}. 
Explicit formulas of the same constants for some 
three-dimensional domains can be found in papers \cite{Berard1980} and 
\cite{HoshikawaUrakawa2010}.
%--------------------------------------------------------------------------------------%

%--------------------------------------------------------------------------------%
The Poincar\'{e}-type inequalities also hold for functions \linebreak
%with zero mean 
%traces on the boundary or a measurable part of it. 
%Let
%%
%\begin{equation}
    %\widetilde{H}^1(\Omega, \Gamma) :=
    %\Big\{ u \in H^1(\Omega)\,  \big | \,{ \big \{  u \big\} }_{\Gamma} = 0 \;\Big\},
    %\label{eq:space-with-boundary-mean}
%\end{equation}
%
%then 
%For any 
$w \in \widetilde{H}^1(\Omega, \Gamma) :=
    \big\{ u \in H^1(\Omega)\,  \big | \,{ \{  u \} }_{\Gamma} = 0 \;\big\}
$, where $\,{  \{  u \} }_{\Gamma} := \tfrac{1}{|\Gamma|} \Int_{\Gamma} w \ds$, i.e., 
%
\begin{alignat}{2}
    \|u\|_{\L{2} (\Omega)} & \leq C^{\mathrm{p}}_{\Gamma} \|\nabla u\|_{\L{2} (\Omega)}, 
		\label{eq:c-omega-poincare-type-inequality}\\
		\|u\|_{\L{2} (\Gamma)} & \leq C^{\mathrm{Tr}}_{\Gamma} \|\nabla u\|_{\L{2} (\Omega)}.
    \label{eq:c-gamma-poincare-type-inequality}
\end{alignat}
%
%--------------------------------------------------------------------------------------%
The exact values of $C^{\mathrm{p}}_{\Gamma}$ and $C^{\mathrm{Tr}}_{\Gamma}$ on right 
triangles, rectangles, and parallelepipeds can be found in \cite{NazarovRepin2014}. 
%We recall the constants for simplices in $\Rtwo$, which are used as the references in the estimates 
%of the constants presented in \cite{RefArxivMatculevichRepin2015}. 
%
We consider below mainly two reference cases in $\Rtwo$:
triangle $\T:= {\rm conv} \big\{ (0, 0), (0, h), (h, 0) \big\}$ and 
$\Gamma := \big \{ x_2 = 0, \, x_1 \in [0, h] \big \}$, and corresponding 
constants  
%
%\begin{equation}
$\CPT := \tfrac{h}{\zeta_0}, \quad \mathrm{and} \quad 
\CGT := \left(\tfrac{h}{\hat{\zeta}_0 \, \tanh({\hat{\zeta}}_0)}\right)^{\rfrac{1}{2}}$,
%\label{eq:exact-cp-cg-t-leg}
%\end{equation}
%
where $\zeta_0$ and $\hat{\zeta}_0$ are the unique roots of the equations 
$z \cot(z) + 1 = 0$ and $\tan(z) + \tanh(z)  = 0$ in $(0, \pi)$, respectively, and 
%
%--------------------------------------------------------------------------------------%
simplex $\T:= {\rm conv} \big\{ (0, 0), (0, h), \big(\frac{h}{2}, \frac{h}{2}\big) \big\}$, 
with \linebreak
$\Gamma := \big \{ x_2 = 0, \, x_1 \in [0, h] \big \}$, which are characterized 
by
%
%\begin{equation}
$\CPT := \tfrac{h}{2 \zeta_0}$ and \linebreak $
\CGT := \big(\tfrac{h}{2}\big)^{\rfrac{1}{2}}$.
%\label{eq:exact-cp-cg-t-hyp}
%\end{equation}
%

Finally, the classic trace inequality reads as follows
%
\begin{equation}
	\| u \|_{\L{2}(\Gamma)} \leq \Ctr{\Gamma} \| \, u \,\|_{\H{1}(\Omega)} \,,
	\qquad \forall u \in \C{1}(\overline{\Omega}).
	\label{eq:trace-inequality}
\end{equation}

\hiddensubsection{Sobolev spaces in the space-time cylinder}
\label{ssec:sobolev-spaces-Qt}

\noindent
%--------------------------------------------------------------------------------------%
Let $Q_T := \Omega \times ]0, T[$ denote the space-time cylinder with given $\Omega$ 
and time interval $]0, T[$, $0 < T < +\infty$. We denote 
$S_T = \partial \Omega \times [0, T]$ as a lateral 
surface of $Q_T$. Below, we introduce the Sobolev spaces of functions defined on $Q_T$ as 
they are presented in \cite{Ladyzhenskaya1985, Ladyzhenskayaetall1967}. The space 
$\L{2}(Q_T)$ contains square-integrable functions in the cylinder $Q_T$ and it is equipped 
with the norm $\| \cdot \|_{\L{2}(Q_T)} := (\cdot, \cdot)^{\rfrac{1}{2}}_{\L{2}(Q_T)}$.
We generalize the notation by denoting the space $\H{s, k}(Q_T)$ as 
%
\begin{alignat*}{2}
\H{s, k}(Q_T) \!:= \!\Big\{ u \in \L{2}(Q_T) \; \mid \;
                         D^\alpha u \in \L{2}(Q_T), \; |\alpha| \leq s,\;
												 \partial_t^\beta u \in \L{2}(Q_T), \, 1 \leq \beta \leq k \Big\}
%\label{eq:H-st}
\end{alignat*}
%
equipped with the norm 
%
\begin{equation*}
\|u\|^2_{\H{s, k}(Q_T)} 
:= \IntQT \bigg( \Sum_{|\alpha| \leq s} | D^\alpha u(x, t) |^2 
                + \Sum_{1 \leq \beta \leq k} |\partial^{\beta}_t u(x, t)|^2 \bigg) \dxt.
%\label{eq:norm-H-sk}
\end{equation*}
%
The most typical examples are $\H{1, 0}(Q_T)$ and $\H{1, 1}(Q_T)$.
In \cite{Ladyzhenskaya1985}, the same spaces are denoted by, e.g., $W_{2}^{1, 0} (Q_T)$
and $W_2^{1, 1} (Q_T)$.
%--------------------------------------------------------------------------------------%
%
%The space $\H{1, 0}(Q_T)$ is defined as
%%
%\begin{equation}
%\H{1, 0}(Q_T) := \bigg\{ u \in \L{2}(Q_T) \; \mid \; 
                         %\nabla u \in \Big[ \L{2}(Q_T) \Big]^d \bigg\}
%\label{eq:H-10}
%\end{equation}
%%
%equipped with norm 
%%
%\begin{equation}
%\|u\|_{\H{1, 0}(Q_T)} := \IntQT \big(|u(x, t)|^2 + |\nabla u(x, t)|^2\big) \dxt,
%\label{eq:norm-H-10}
%\end{equation}
%%
%and the space $\H{1, 1}(Q_T)$ as
%%
%\begin{equation}
%\H{1, 1}(Q_T) := \bigg\{ u \in \L{2}(Q_T) \; \mid \; 
                         %\nabla u \in \Big[ \L{2}(Q_T) \Big]^d, 
												 %\partial_t u \in \L{2}(Q_T) \bigg\}
%\label{eq:H-11}
%\end{equation}
%%
%with 
%%
%\begin{equation}
%\|u\|_{\H{1, 1}(Q_T)} := 
%\IntQT \big(|u(x, t)|^2 + |\nabla u(x, t)|^2 |\partial_t u(x, t)|^2 \big) \dxt.
%\label{eq:norm-H-11}
%\end{equation}
%%
%%%--------------------------------------------------------------------------------%%
%Here, $\nabla = \nabla_x$ and $\partial_t$ demote weak spatial gradient and time 
%derivative, respectively. Analogously to the definitions above, we can define 
%Sobolev space
%%
%\begin{equation}
%\H{0, 1}(Q_T) := \bigg\{ u \in \L{2}(Q_T) \; \mid \; 
												 %\partial_t u \in \L{2}(Q_T) \bigg\}
%\label{eq:H-01}
%\end{equation}
%%
%with corresponding norm. 
%
%--------------------------------------------------------------------------------------%
Furthermore, the Sobolev spaces with Dirichlet boundary $S_D \subset S_T$ (with 
assigned load $u_D$ on it) are denoted by
%
\begin{alignat}{2}
\HD{s, k}{u_D}(Q_T) & := \Big\{ u \in \H{s, k}(Q_T) \; \mid \; 
                             u = u_D \quad {\rm on} \quad S_D \Big\}.
\label{eq:H-sk-dirichlet}
\end{alignat}

\section{Bochner spaces}
%\label{sec:bochner-spaces}

%--------------------------------------------------------------------------------------%
Consider the Bochner spaces as an alternative tool for the analysis of parabolic 
\linebreak I-BVPs. 
Let $\{H, (\cdot, \cdot)_H\}$ and $\{V, (\cdot, \cdot)_V\}$ be a Hilbert space. 
%Then the Bochner space $Y(a, b; H)$ 
%can be defined as a linear space 
%%of all equivalence classes of Lebesgue-measurable 
%%functions $u: (a, b) \rightarrow H$ such that 
%%
%\begin{equation*}
%Y(a, b; H) := \Big\{ u: (a, b) \rightarrow H \;\,| \;\,
                     %\| u\|_{Y(a, b; H)} < +\infty \Big\}.
%%\label{eq:Y-ab-H}
%\end{equation*}
%
%--------------------------------------------------------------------------------------%
The Bochner spaces $\L{p}(a, b; V)$, 
$p \in [1, +\infty[$, are the most regularly used. They consist of 
measurable functions $u: ]a, b[ \rightarrow V$ for which norm reads as   
%
\begin{equation*}
\|u\|_{\L{p} (a, b; V)} := \Big( \Int_a^b \| u(\cdot, t)\|^p_{V} \dt \Big)^{\rfrac{1}{p}} < +\infty.
%\label{eq:norm-Lp-ab-H}
\end{equation*}
%
%is finite. 
For $p = \infty$, we obtain the Bochner space equipped with the norm
%
\begin{equation*}
\|u\|_{\L{\infty} (a, b; V)} 
:= {\rm ess} \Sup_{t \in (a, b)} \| u(\cdot, t)\|_{V} < + \infty.
%\label{eq:norm-linfty-ab-H}
\end{equation*}
%
%--------------------------------------------------------------------------------------%
Furthermore, we define $C([a, b]; H)$ as the space of functions 
$u: [a, b] \rightarrow H$ continuous at every $t \in [a, b]$ with the norm 
%
\begin{equation*}
\|u\|_{C ([a, b]; H)} 
:= \max\limits_{t \in [a, b]} \| u(\cdot, t)\|_{H}.
%\label{eq:norm-C-ab-H}
\end{equation*}
%
%--------------------------------------------------------------------------------------%
Infinitely differentiable functions are denoted by $\C{\infty}([a, b]; H)$ and 
$\Co{\infty} ([a, b]; H)$ (in case the functions have compact support on $(a, b)$).

For the treatment of parabolic I-BVPs, we consider $\L{2} (0, T; V)$ with \linebreak
$V = \H{1}(\Omega)$ (or $V = \Ho{1}(\Omega)$). Since $V$ is a 
Hilbert space, then $\L{2} (0, T; V)$ is also a Hilbert space. 
The generalized weak derivative of  $u \in \L{2} (0, T; V)$ with respect to time is 
denoted by $\partial_t u \in \L{2} (0, T; V^*)$, satisfying  
%
\begin{equation*}
\Int_0^T u(t) \partial_t \varphi (t) \dt 
= -\Int_0^T \partial_t u(t) \varphi (t) \dt, \quad \forall
\varphi \in C^{\infty}_0 ([0, T]; H).
%\label{eq:gen-weak-derivative-ut} 
\end{equation*}
%
%--------------------------------------------------------------------------------------%
%--------------------------------------------------------------------------------------%
For separable $V$ and $H$, the Gelfand triple (or evolution triple) \linebreak
$V \hookrightarrow H \hookrightarrow V^*$ holds. Then, 
$V^*$ is a Hilbert space. The most commonly used triples are  
$\H{1}(\Omega) \hookrightarrow \L{2}(\Omega) \hookrightarrow (\H{1}(\Omega))^*$ and 
$\HD{1}{0}(\Omega) \hookrightarrow \L{2}(\Omega) \hookrightarrow \H{-1}(\Omega)$.

To study the solvability of the parabolic I-BVPs, we define the Bochner space 
%
%\begin{equation*}
$W(0, T) := \Big\{u(t) \in \L{2} (0, T; V) \, | \,
         \partial_t u(t) \in \L{2} (0, T; V^*) \Big\}$
%\label{eq:W-0T}
%\end{equation*}
%
equipped with the norm
%
\begin{equation*}
\|u\|_{W(0, T)} 
:= \bigg(\Int_0^T \Big( \| u(\cdot, t)\|^2_{V} 
                         + \| \partial_t u(\cdot, t)\|^2_{V^*} 
									\Big) \dt \bigg)^{\rfrac{1}{2}} < \infty.
%\label{eq:norm-W-0T}
\end{equation*}
%
The Gelfand triple implies that $W(0, T)$ is a Hilbert space. 
Moreover, we have the continuous embedding $W(0, T) \hookrightarrow C(0, T; H)$ 
(see, e.g., \cite{Wloka1987} and \cite{Zeidler1990A}).
%--------------------------------------------------------------------------------------%
The formula of integration by parts reads as 
%
\begin{equation*}
\Int_0^T \! \langle \partial_t u(t),  \varphi (t) \rangle_{V^*, V} \dt 
= - \! \Int_0^T \langle \partial_t \varphi (t), u(t) \rangle_{V^*, V} \dt 
+ (u(T),  \varphi (T)) - (u(0),  \varphi (0)),
%\label{eq:int-by-parts} 
\end{equation*}
%
where 
$\partial_t u(t) \in \L{2} (0, T; V^*)$, 
$\varphi(t) \in \L{2} (0, T; V)$, and 
$\partial_t \varphi(t) \in \L{2} (0, T; V^*)$.
%

By comparing the norms of the spaces discussed above, one can see 
how Bochner spaces correlate with Sobolev spaces, e.g., 
%
\begin{equation*}
	\H{1, 0}(Q_T) \cong \L{2}(0, T; \H{1}(\Omega)), \quad
	\HD{1, 0}{0}(Q_T) \cong \L{2}(0, T; \HD{1}{0}(\Omega)).
\end{equation*}
%--------------------------------------------------------------------------------------%
If, in addition, we consider the space $\H{1}(0, T; \L{2}(\Omega))$ with finite norm
%
\begin{equation*}
\|u\|_{\H{1}(0, T; \L{2}(\Omega))} 
:= \bigg(\Int_0^T \Big( \| u(\cdot, t)\|^2_{\L{2}(\Omega)} 
                         + \| \partial_t u(\cdot, t)\|^2_{\L{2}(\Omega)} 
									\Big) \dt \bigg)^{\rfrac{1}{2}},
%\label{eq:norm-H-1-0T}
\end{equation*}
%
%--------------------------------------------------------------------------------------%
then the combination of the norms corresponding to spaces $\H{1, 1} (Q_T)$ and 
$\HD{1, 1}{0} (Q_T)$ (in some literature denoted by $\H{1} (Q_T)$ and 
$\HD{1}{0} (Q_T)$) provides the equivalences 
%
\begin{alignat*}{2}
	\H{1, 1} (Q_T) & \cong  \L{2}(0, T; \H{1}(\Omega)) \, \cap \, 
													\H{1}(0, T; \L{2}(\Omega)), \\
	\HD{1, 1}{0} (Q_T) & \cong  \L{2}(0, T; \HD{1}{0}(\Omega)) \, \cap \, 
													\H{1}(0, T; \L{2}(\Omega)).									
\end{alignat*}
%
Bochner space $W(0, T)$ with $V = \H{1}(\Omega)$ ($V = \HD{1}{0}(\Omega)$) is clearly 
wider than $\H{1, 1} (Q_T)$ ($\HD{1, 1}{0} (Q_T)$) based on evolution triple. 
%
Finally, we introduce, in general form, $\V{s, k} (Q_T)$ and $\VD{s, k}{0}(Q_T)$ 
(following the notation in \cite{Ladyzhenskaya1985}) such that
%
\begin{equation*}
\V{s, k} (Q_T) := \H{s, k}(Q_T) \cap C([0, T]; \L{2}(\Omega)), \;\,
\end{equation*}
%
and 
%
\begin{equation*}
\VD{s, k}{0}(Q_T) := \HD{s, k}{0}(Q_T) \cap C([0, T]; \L{2}(\Omega)). 
\end{equation*}
%
respectively, where $s \geq 0$, $k \geq 0$, equipped with the norm 
%
\begin{equation*}
	\|u\|_{\V{s, k} (Q_T)} 
	:= \max\limits_{t \in [0, T]} \| u(t)\|_{\L{2}(\Omega)} + 
	   \|u\|_{\H{s, k}(Q_T)} < +\infty.
	%\label{eq:norm-tildeH-0T}
\end{equation*}
%

\section{Parabolic initial-boundary value problem}
%\label{sec:parabolic-i-bvp}
%--------------------------------------------------------------------------------------%
In the current section, we present fundamental results on solvability of linear 
parabolic PDEs, which have been thoroughly studied in monographs
\cite{Ladyzhenskaya1985,Friedman1964,Zeidler1990A,Wloka1987}. The nonlinear class 
is considered in the monographs \cite{Ladyzhenskayaetall1967,Zeidler1990B}.
Below, we present the variational formulation of a parabolic I-BVP and discuss the main 
requirements that provide the existence and uniqueness results.

Let $Q_T$ be a space-time cylinder with boundary surface $S_T$ as defined in Section 
\ref{ssec:sobolev-spaces-Qt}. Assume that $\partial \Omega$ consists of two measurable 
non-intersecting parts $\Gamma_D$ and $\Gamma_R$ associated with mixed Dirichlet--Robin 
BC. Therefore, \linebreak $S_T := \partial\Omega \times [0, T] = $ $
\big( \Gamma_D \cup \Gamma_R \big) \times [0, T] = S_D \cup S_R$. 
%
%--------------------------------------------------------------------------------------%
The general parabolic I-BVP reads as follows
%
\begin{alignat}{3}
	u_t - \dvrg p + a(x) \cdot \nabla u + \lambda^2 (x) \, u & =\, f,	      
	& \quad (x, t) \in Q_T,\label{eq:parabolic-equation}\\
  p & =\, A \nabla u, & \quad (x, t) \in Q_T,\label{eq:dual-part}\\
  u(x, 0) & =\, u_0,		
	& \quad x \in \Omega,
	%\label{eq:parabolic-initial-condition}
	\\
  u & =\, 0,		      
	& \quad (x, t) \in S_D,\label{eq:parabolic-dirichlet-bc}\\
  \sigma^2(x) u + p \cdot n & =\, 0,					
	& \quad (x, t) \in S_R,\label{eq:parabolic-robin-bc}
\end{alignat}
%
where $n$ denotes the vector of unit outward normal to $\partial\Omega$, 
%
\begin{equation}
f \in \L{2}(Q_T), \quad  
u_0 \in \L{2}(\Omega).
%, \quad 
%{g(x, t) \in \L{2}\left(0, T; \H{\rfrac12} (\Gamma_R)\right), } 
\label{eq:problem-condition}
\end{equation}
%
%because $f(x, t) \in R1$
%--------------------------------------------------------------------------------------%
We assume that, for almost all 
$x \in \Omega$ and  $t \in ]0, T[$, the operator $A$ is 
symmetric and satisfies condition of uniform parabolicity 
%	
\begin{equation}
\underline{\nu}_A |\xi|^2 \leq A(x, t) \: \xi \cdot \xi \leq \overline{\nu}_{A} |\xi|^2, 
\quad \xi \in \Rd,\quad 0 < \underline{\nu}_A \leq \overline{\nu}_A < \infty.
\label{eq:operator-a}
\end{equation}
%
%--------------------------------------------------------------------------------------%
Henceforth, we use the notation
%
\begin{equation*}
  \| \, \tau \, \|^2_A := \Int_\Omega A \tau \cdot \tau \dx, \quad 
	\| \, \tau \, \|^2_{A^{-1}} := \Int_\Omega A^{-1} \tau \cdot \tau \dx.
	%\label{eq:norms}
\end{equation*}
%
The functions $a$ and $\lambda$, presenting the convection and reaction, respectively,  
as well as $\sigma$ satisfy the following conditions for a.a $t \in ]0, T[$
%
\begin{alignat}{2}
	a \in L^{\infty} (\Omega, \Rd), \quad \dvrg \,a \in L^{\infty} (\Omega), 
	\quad |a| \leq \overline{a}, \nonumber\\
	\lambda \in L^{\infty} (\Omega), \quad |\lambda| \leq \overline{\lambda}, \nonumber\\
	\sigma \in L^{\infty} (\Omega), \quad |\sigma| \leq \overline{\sigma}.
	\label{eq:coefficients-condition}
\end{alignat}
%
%--------------------------------------------------------------------------------------%
%
After multiplying (\ref{eq:parabolic-equation}) by a test function 
$\eta \in \HD{1, 1}{0}(Q_T)$, we arrive at the generalized formulation of 
(\ref{eq:parabolic-equation})--(\ref{eq:parabolic-robin-bc}): 
find $u(x, t) \in \VD{1, 0}{0}(Q_T)$ (cf. \eqref{eq:H-sk-dirichlet}) 
satisfying the integral identity
%
\begin{multline}
	\Int_{Q_T} \Big( 
	A \nabla{u} \cdot \nabla{\eta} +  
	a \cdot \nabla u  \eta + \lambda^2 u \eta -  u \eta_t \Big) \dxt
	+ \Int_{S_R} \sigma^2 u \eta \dst \\
	+ \Int_{\Omega} \big( (u\eta)(x, T) - (u\eta)(x, 0) \big) \dx =	 
	\Int_{Q_T} f \eta \dxt , 
	\quad \forall \eta \in \HD{1, 1}{0}(Q_T).
	\label{eq:generalized-statement}
\end{multline}
%
According to \cite[Theorem 3.2]{Ladyzhenskaya1985}, 
the generalized problem (\ref{eq:generalized-statement}) has a solution in 
$\VD{1, 0}{0}(Q_T)$ and it is unique in $\HD{1, 0}{0}(Q_T)$, 
provided that conditions (\ref{eq:problem-condition}), 
(\ref{eq:operator-a}), and (\ref{eq:coefficients-condition}) hold. In the problem with 
only Robin BC, in order to provide the uniqueness of the solution additional conditions
on coefficients  
%
\begin{equation*}
|\partial_t a| \leq \widetilde{a}, \quad
|\partial_t \lambda| \leq \widetilde{\lambda}, \quad 
|\partial_t \sigma| \leq \widetilde{\sigma},
\end{equation*}
%
%--------------------------------------------------------------------------------------%
must be imposed. The a priori stability estimate 
%for the solution measured by the norm defined in 
%\eqref{eq:norm-tildeH-0T}
%
\begin{equation}
\| u \|_{\VD{1, 0}{0} (Q_T)} 
\leq C \Big( \| f\|_{\L{2}(Q_T)} + \|u_0\|_{\L{2}(\Omega)} \Big)
\label{eq:stability-estimate}
\end{equation}
%
holds with a positive constant $C$ dependent only on characteristics of $Q_T$ but
independent of $f$ and $u$. The estimate
(\ref{eq:stability-estimate}) provides the continuity of the mapping 
$\mathcal{M}: \{f, u_0\} \mapsto u$, where $\mathcal{M}:$
$\L{2}(Q_T) \times \L{2}(\Omega)\mapsto \VD{1, 0}{0} (Q_T)$.

The solvability results can be formulated in Bochner spaces. We consider the 
simplest case, where $A = I$ , 
$a(x) = 0$, $\lambda(x) = 0$, and $S_T = S_D$. According to 
\cite{Wloka1987, Zeidler1990A}, if $H$ and $V$ are given separable Hilbert spaces
satisfying evolution triple 
$V \hookrightarrow H \hookrightarrow V^*$, $f \in \L{2}(0, T, V^*)$, 
$u_0 \in H$, then the generalized problem
%
\begin{equation*}
	\Int_{0}^{T} \langle u_t (t), v \rangle_{V^*, V} \dt + 
	\Int_{0}^{T} \nabla{u}(t) \cdot \nabla{v} \dxt = 
	\Int_{0}^{T} \langle f(t), v \rangle_{V^*, V} \dxt,
	%\label{eq:generalized-statement-in-bochner}
\end{equation*}
%
%--------------------------------------------------------------------------------------%
that holds for all $v \in V$ and a.a.$t \in ]0, T[$, 
has a unique solution in $W(0, T)$, which depends continuously on $f$ and $u_0$. 
By increasing the regularity on $u_0$ and $f$, one can get higher regularity of 
the exact solution. The 
problems with inhomogeneous BCs, e.g., $u_D$ in \eqref{eq:parabolic-dirichlet-bc} 
and $g$ in \eqref{eq:parabolic-robin-bc}, can be treated in the same manner, 
following the spirit of \cite{Wloka1987}.

\section{Fixed point iterations}
%\label{sec:iteration-methods}

First, we present the main idea of the fixed-point iterations approach.
%--------------------------------------------------------------------------------------%
Consider the following general problem: find $u$ in a Hilbert space $V$ such that
%
\begin{equation}
    u = \operL u + b,
    \label{eq:equivalent-form-of-system-problem}
\end{equation}
%
where $\operL: V \rightarrow V$ is a bounded operator and $b \in V$. One of the ways to 
solve \eqref{eq:equivalent-form-of-system-problem} is to apply the iteration procedure
%
\begin{equation*}
  u_k = \operL u_{k-1} + b, \quad u_0 \in V, \quad k = 1, \ldots
	%\label{eq:approximation-of-fixed-point-equation}
\end{equation*}
%
which generates an infinite sequence $\{u_k\}_{k = 1}^\infty$. If $\operL$ is the 
{\em $q$-contractive operator} on a closed nonempty set $S \subset V$, i.e.,
%
\begin{equation}
   \|\operL w - \operL v\|_V \,\leq\, q \, \|w - v\|_V, \quad q \in (0, 1), 
		\quad \forall w, v \in S,
    \label{eq:contractivity}
\end{equation}
%
then, by using (\ref{eq:contractivity}), it is easy to show that 
$\{u_k\}_{k = 1}^\infty$ converges to a fixed point $u$ (see, e.g., 
\cite{Banach1922,Collatz1964,KolmogorovFomin1975,Istratescu1981,Zeidler1986}).

\hiddensubsection{The Picard--Lindel\"{o}f method}
\label{ssec:pl-method}
%--------------------------------------------------------------------------------------%
We consider a Cauchy problem
%
\begin{equation}
  \tfrac{du}{dt} = \varphi(u(t),\;t),\quad u(t_0) = a_0, \quad 
	t \in [t_0 - \varepsilon, t_0 + \varepsilon]
  \label{eq:cauchy-problem}
\end{equation}
%
with (scalar- or vector-valued) solution $u(t)$. Assume that the function 
$\varphi(u(t),\:t)$ is uniformly Lipschitz continuous with respect to $u$ (i.e., 
Lipschitz constant can be selected independent of $t$) and continuous in $t$. The 
existence and uniqueness of continuously differentiable $u(t)$ on 
$[t_0 - \varepsilon, t_0 + \varepsilon]$, 
$\forall \varepsilon > 0$, follows from the \PL theorem and the Picard's existence 
theorem (or Cauchy--Lipschitz theorem) (see \cite{CoddingtonLevinson1972,Lindelof1894}).
%
Unlike the \PL theorem, the Peano existence theorem \cite{Peano1886} shows only existence, not uniqueness, but imposes weaker requirements on $\varphi$ (only the continuity with respect to $t$).

The \PL method represents \eqref{eq:cauchy-problem} in the 
integral form
%
\begin{equation}
    u(t) = \Int_{t_0}^{t}\varphi(u(s),\;s) \ds + a_0.
    \label{eq:integral-form-of-problem}
\end{equation}
%
The exact solution of (\ref{eq:integral-form-of-problem}) is a fixed point
that is approximated by the iterative method
%
\begin{equation}
	u_j = \Tau u_{j - 1} + a_0, 
	\quad \Tau u := \Int_{t_0}^{t}\varphi(u(s), s) \ds
  \label{eq:iterative-form-of-problem}
\end{equation}
%
provided that $\Tau: V \rightarrow V$ satisfies \eqref{eq:contractivity} on 
$[t_0 - \varepsilon, t_0 + \varepsilon]$.

%The Carath\'{e}odory's existence theorem is formulated under weaker conditions on $\varphi$. It is also interesting to remark that although these conditions are only sufficient, there also exist necessary and sufficient conditions for the solution of an initial value problem to be unique, such as Okamura's theorem. [4]
