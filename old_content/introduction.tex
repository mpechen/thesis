\chapter{Introduction}
% - short history
% - importance of the problem of concept drift and change detection
% - examples of the related real problems 
% - what is in this document
% http://www.itl.nist.gov/div898/handbook/index2.htm
% https://en.wikipedia.org/wiki/Exchangeable_random_variables
% http://www.itl.nist.gov/div898/handbook/index.htm !!!

% I thank God for his provision, protection, patience and guidance through my career. 
% from:http://www.natemarquardt.com/2017/12/farewell-to-ufc.html?m=1
% https://www.reddit.com/r/MMA/comments/7moztm/nate_marquardt_retires_here_is_his_statement/?st=jbrrqma0&sh=2aff771c

\subsection{Importance of the change-detection problem}
Changepoint is a time moment when parameters of a probability distribution generating observations in a time series change.
\textit{`Changepoint' is a phenomenon when statistical properties of the data stream change \textbf{significantly} according to some predefined criteria.}
%\begin{figure}
%    \includestandalone[height = 0.20\textwidth]{./tikz/change_example.tex}
%\end{figure}
% Or, in other words, changepoint is a time moment when statistical properties of the time series changes sign
Change detection is the process of identifying differences in the state of an object or phenomenon by observing it at different times~\cite{Singh1989}.
%
Many practical questions across industry related to the changedetection arise
\begin{itemize}
    \item When is it a time to inject an additional dose of propofol into the patient undergoing surgery? cite
    \item Imagine a car on a slippery road - when is an optimal time moment to activate an Anti-lock braking system (ABS)?
    \item When to inspect machinery because technical indicators being read from the attached sensory network are degraded.
\end{itemize}

The are many applications:
\begin{itemize}
    \item Fault detection and monitoring
    \item Vibration monitoring of mechanical systems
    \item Automatic depth of anesthesia control 
    \item DM: Predictive modeling under concept drift
\end{itemize}

%%% start
A good \textit{online} change detector should:
\begin{itemize}
        \item detect all changes within an acceptable time lag (high true positive rate (TPR)),
        \item be robust to noise by not raising false alarms (low false positive rate (FPR)).
\end{itemize}

\def \sensp {\pmb{s}}
Detector has a sensitivity parameter $\sensp$:
\begin{itemize}
    \item If $\sensp$ is (too) high: all changes are detected but some outliers misclassified as changes.
    \item If $\sensp$ is (too) small: we skip outliers but  change detection lag increases.
\end{itemize}        
next
%\begin{figure}
%    \includestandalone[height = 0.7\textheight]{./tikz/change_detectors_behavior}
%\end{figure}
%%% end

Change detection in remotely-sensed data from Earth-orbiting satellites
is useful 

Adaptivity to the changing environment implies awareness about the changes in this environment.

?Connection to the streaming algorithms~\cite{}

Sensor generated data are ubiquitous.
A lot of data is generated by sensors embedded into mobile devices, and etc.

In medicine observations readed from patient  In needs to be monitored for changes for ..

In telecom

In the Data Mining area the problem is known as a ``concept drift'' which is a phenomenon when the joint probability distribution of the input data and the target variable being predicted changes over time~\cite{gama2014survey}.

- for resource planning.

- for vibration monitoring in constructions.
E.g. bridges - if vibration is increasing then the bridge must be checked 

In medicine (in anesthesia control task) if anesthesia depth is below or above predefined thresholds an action must be taken.

Sources are mobile phones applications, social networks, wearable sensors in mobile phones.

There are a lot of sensors generating data continuously.
There are many applications for the changedetection methods in many application areas. 
in 

Changedetection is crucially important for decision making and forecasting tasks.

To do time series forecasting a stationary PDF generating data is often assumed.
Once underlying probability distribution changes predictive model must be updated.

Presence of the changepoint triggers decision making process.

\cd is a clue for analyzing large volumes of data on-line 

- industry 
- medicine 
- telecom~\cite{gama2010knowledge}



%% p.17
Incremental learning algorithms must adapt to the concept
drift~\cite{gama2010knowledge}.


a lot of data are being generated 
needs to be analyzed 
dynamic models 

predictive models
change detection ..

%!!! http://www.itl.nist.gov/div898/handbook/index.htm
% http://www.itl.nist.gov/div898/handbook/pmc/section1/pmc12.htm
%%
%% Multivariate control charts Underlying concepts The underlying concept of statistical process control is based on a
%% comparison of what is happening today with what happened previously. We take a snapshot of how the process typically
%% performs or build a model of how we think the process will perform and calculate control limits for the expected
%% measurements of the output of the process. Then we collect data from the process and compare the data to the control
%% limits. The majority of measurements should fall within the control limits. Measurements that fall outside the control
%% limits are examined to see if they belong to the same population as our initial snapshot or model. Stated differently,
%% we use historical data to compute the initial control limits. Then the data are compared against these initial
%% limits. Points that fall outside of the limits are investigated and, perhaps, some will later be discarded. If so, the
%% limits would be recomputed and the process repeated. This is referred to as Phase I. Real-time process monitoring, using
%% the limits from the end of Phase I, is Phase II.

The roots of the change detection problem are in the field of Statistical process control (SPC) theory developed in
1920s~\cite{TartakovskySeq} by Walter A.  Shewhart~\cite{shewhart1931economic},~\cite{shewhart1931economic} at Bell
Labaratories.  He developed the control chart~\cite{shewhart1926quality} and the concept of a state of statistical
control.

%
Later it became a part of more general problem - concept drift detection(Joao Gama).

Change detection is an important problem in the field
of Structural Health Monitoring~\cite{MiaoMultiSensor}
\footnote{\href{http://liacs.leidenuniv.nl/~csinfra/}{InfraWatch project}}.

In the energy sector changedetection is used to

In the telecom indistry
to 

%
%%..... BEGIN AIRPLANNE QUOTES .....%%%
%% From  http://www.datasciencecentral.com/profiles/blogs/that-s-data-science-airbus-puts-10-000-sensors-in-every-single
%% The current A350 model has a total of close to 6,000 sensors across the entire plane and generates 2.5 Tb of data per day, while the newer model – expected to take to he skies in 2020 – will capture more than triple that amount.
%% In an industry as driven by technology as the aviation industry, it’s hardly surprising that every element of an aircraft’s performance is being monitored for the potential to make adjustments which could save millions on fuel bills and, more importantly, save lives by improving safety.
%% Engines are equipped with sensors capturing details of every aspect of their operation, meaning that the impact of humidity, air pressure and temperature can be assessed more accurately. It is far cheaper for a company to be able to predict when a part will fail and have a replacement ready, than to wait for it to fail and take the equipment offline until repairs can be completed.
%%..... END AIRPLANNE QUOTES .....%%%


In the aviation industry Airbus A350 is equipped with 6000 sensors across the
entire plane and generates 2.5Tb of data per day.
Newer model Airbus A380 will be equipped with 10.000 sensors in each wing.  
Sensor recordings allow increase operational efficiency 

Sensors are devices measuring physical properties evolving over time.
fields of civil engineering, windmills, aviation 
% http://liacs.leidenuniv.nl/~csinfra/#dataset - papers

Examples of the sources of the sensor generated data
\begin{itemize}
	\item Airplanes
	\item ISP providers
	\item Car engines
    \item biomedical applications: to monitor pulse rate, respiration in order to improve sleep quality 
    \item EEG monitoring 
    \item Unusual sequences of credit card transactions (NN lec.)
    \item Unusual pattern of sensor readings in a nuclear power plant (NN lec.)
\end{itemize}
% About sensors good: 
% Modeling Sensor Dependencies between Multiple Sensor Types. Miao, S., Vespier, U., Vanschoren, J., Knobbe, A. Paper in Proceedings Benelearn 2013.
% http://liacs.leidenuniv.nl/~csinfra/#dataset
%\section{Industrial data streams}
%\section{On-line and off-line settings}
%\section{Change detection problem}
%\section{Context awareness}
%\section{Prediction of the changes using contextual information}

\subsection{Contribution}

In the thesis we develop the framework  
