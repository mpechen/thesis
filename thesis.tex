\documentclass[licentiate,utf8,lot,loar,lof,shortloft,index]{jydiss}
%\documentclass[licentiate,latin9,loa,lot,lof,shortloft,captiondot]{jydiss}
%\documentclass[licentiate,finnish,latin9,loa,lot,lof]{jydiss}
\usepackage{algorithm}% http://ctan.org/pkg/algorithms
\usepackage{algpseudocode}% http://ctan.org/pkg/algorithmicx
\usepackage{listings}
\usepackage{hyperref}
\usepackage{enumitem}
\usepackage{array}
\usepackage{amsmath}
\usepackage{amssymb}
\usepackage{tikz,ulem}
\usepackage{graphics}
\usepackage{standalone}
\usepackage{amsthm}
\usepackage{breakcites}

%\usepackage{ltexpprt}
%\algtext*{EndWhile}% Remove "end while" text
%\algtext*{EndIf}% Remove "end if" text
%\algtext*{EndFor}% Remove "end if" text
\usepackage{cite}
%\usepackage{graphicx}
\usepackage{soul} % to cross text
\usepackage{enumitem}




\newtheorem{definition}{Definition}
\newtheorem{theorem}{Theorem}

\newcommand*{\LargeCdot}{\raisebox{-0.5ex}{\scalebox{1.8}{$\cdot$}}}
\newlist{WithAxioms}{enumerate}{1}
\setlist[WithAxioms]{label=Axiom \arabic*:}


\newcommand{\T}{\mathcal{T}} % index set
\newcommand{\Collect}[1]{\langle #1 \rangle} % vector/sequence notation
%\newcommand{\Ed}[1]{\colorbox{green!20}{#1}} % highlight editions
\newcommand{\Ed}[1]{{\color{blue!100}#1}} % highlight editions
\newcommand{\Eq}[1]{Eq.(\ref{#1})} % ref equations as Eq.(reference)
\newcommand{\lbl}[1]{(#1)} % to denote output labels of classifier
%\newcommand{\Event}[2]{\mathtt{E}(#1)^{#2}} % to denote CdEs (+ or -)
%\newcommand{\Event}[2]{\mathbf{e}(#1)^{#2}} % to denote CdEs (+ or -)
\newcommand{\Event}[2]{\mathbf{e}_{#1}^{#2}} % to denote CdEs (+ or -)


% from blpa 
\def \BD   {\textbf{BD }}
\def \PCCF {\textbf{PCCF }} 
\def \PCCFWithoutSpace {\textbf{PCCF}} 
\def \LEAP {\textbf{LEAP }} 

%\newcommand{\Collect}[1]{\langle #1 \rangle} % vector/sequence notation
%\newcommand{\Event}[2]{\mathbf{e}_{#1}^{#2}} % to denote CdEs (+ or -)
%\newcommand{\lbl}[1]{(#1)} % to denote output labels of classifier
%\newcommand{\T}{\mathcal{T}} % index set
%\newcommand{\Eq}[1]{Eq.(\ref{#1})}

\newcommand{\GammaDistr}{\text{Gamma}}

\DeclareMathOperator*{\argmax}{\argmax}

\newcommand{\PccfII}[2]{\textit{Pccf}(#1,#2)}
\newcommand{\PccfI}[1]{\textit{Pccf}(#1)}

%\newcommand*{\LargeCdot}{\raisebox{-0.5ex}{\scalebox{1.8}{$\cdot$}}}

\def\mucommon        {\tilde{\mu}}
\def\muzerocommon    {\tilde{\mu}_0}
\def\sigmacommon     {\tilde{\sigma}}
\def\xcommon         {\tilde{x}_i}
\def\taucommon       {\tilde{\tau}}
\def\kappazerocommon {\tilde{\kappa}_0}
\def\kappacommon     {\tilde{\kappa}}
\def\alphazerocommon {\tilde{\alpha}_0}
\def\alphacommon     {\tilde{\alpha}}
\def\betazerocommon  {\tilde{\beta}_0}
\def\betacommon      {\tilde{\beta}}


\title{Predictive analytics with online changedetection in data streams}
% \entitle{foo}
\setauthor{\rm Alexandr}{\rm Maslov}

%----------------------------------------------------------------------------------%
\abstract{
    This is an English abstract.
}
%----------------------------------------------------------------------------------%

\keywords{
  Change detection, 
  error correction, \\
}

\people{
\item[Author]
  \textit{Alexandr Maslov} \\
    Department of Mathematics and Computer Science\\
    Eindhoven University of Technology (TU/e) \\
    and\\
    Department of Mathematical Information Technology\\
    University of Jyv\"{a}skyl\"{a} (JYU)\\
    Finland

    \item[Supervisors] 
      \textit{Prof. Dr. Mykola Pechenizkiy}\\[0.3em]
      Department of Computer Science\\
      Department of Computer Science\\
      Eindhoven University of Technology (TU/e)\\
      The Netherlands\\

      \textit{Prof. Dr. Tommi K\"{a}rkk\"{a}inen}\\[0.3em]
      Department of Mathematical Information Technology\\
      University of Jyv\"{a}skyl\"{a}\\
      Finland

    \item[Reviewers] XXX
    XXX
%	\item[Reviewers] 
%		
%		\textit{Prof. Dr. Roland Glowinski}\\[0.3em]
%		University of Houston \\
%		Department of Mathematics \\
%		Houston, TX \\
%		USA
%		
%		\textit{Prof. Dr. Ulrich Langer}\\[0.3em]
%		Institute of Computational Mathematics \\
%		Johann Radon Institute for Computational and \\
%		Applied Mathematics (RICAM) \\
%		Austrian Academy of Sciences (\"{O}AW) \\
%		Austria
    % \item[Opponent] XXX
}
\isbn[nid.]{123-456-78-9012-3}
\isbn[PDF]{345-678-90-1234-5}
%\makeindex
% Tommi K\"{a}rkk\"{a}inen
%        \email{tommi.karkkainen@jyu.fi}
%       \affaddr{Dept. of Mathematical IT,}\\
%       \affaddr{University of Jyv\"{a}skyl\"{a}}\\
%       \affaddr{P.O. Box 35, FIN-40014}\\
%       \affaddr{Finland}\\
%       \email{tommi.karkkainen@jyu.fi}
% Mykola Pechenizkiy
%        \affaddr{Dept. of CS, TU Eindhoven}\\
%\affaddr{P.O. Box 513, NL-5600MB}\\
%\affaddr{the Netherlands}\\
%\email{a.maslov@tue.nl,\\ m.pechenizkiy@tue.nl}
%%%%%..... END JYU TEMPLATE



\makeindex
\begin{document}
\preface
\index{aaaaa} aaaaa
\acknowledgements
\begin{notations}
\notation{$:=$}{equals by definition}
\end{notations}
\mainmatter

%===============================================================================%
%                              INTRODUCTION 
%===============================================================================%
\chapter{Introduction}
\chapter{Concept drift problem}
%===============================================================================%
Concept drift~\cite{schlimmer1986incremental,gama2014survey} is a
phenomenon when relation between the input data and the target variable changes
over time~\cite{gama2014survey}.
Formally can be defined~\cite{gama2014survey} as Equation\ref{eq:concept_drift} 
\begin{equation}\label{eq:concept_drift}
\exists X: p_{t_0}(X,y) \neq  p_{t_1}(X,y)
\end{equation}
Adaptive learning refers to updating model online to react to concept
drifts~\cite{gama2014survey}.
%===============================================================================%
\chapter{Change detection problem}
Machine Learning models learn the relation between input data $X$ and target
variable $y$ by approximating a joint distribution $P(X,y)$.  Model performance
degrades when learned underlying data distribution changes.  Therefore concept
drift can be detected by monitoring change points in model's output performance
statistics.
%===============================================================================%
On-line change detection in time series data is an old practical problem with
the roots in the problem of statistical quality
control~\cite{basseville1993detection}, ~\cite{NISTbook}.  Walter A. Shewhart
invented control charts in 1924 while working on the problem of statistical
quality control to improve reliability of telephone transmission systems.
%===============================================================================%
Quality control example: $X$ is a set of sensor readings and $y=good$ is a
quality of the produced item.
%===============================================================================%
Offline and online.
In offline learning all training data is available during training.
In online learning the data is processed sequentially from data streams.
Model is being updated as more data arrives.
Data evolve over time in dynamically changing environments.
\section{PELT method}
Offline detector used in online settings in ~\cite{marrero2013aclac}.

\section{Bayesian detector}
\cite{adams2007bayesian}

\section{CUSUM detector}

%===============================================================================%
%                               RECURRENCY
%===============================================================================%
\chapter{Predictability And Recurrency}
Adaptive learning in concept drift.
Predictability of events in the data stream.
~\cite{feller2008introduction}
Sums of independent random variables.
Recurrency is a form of predictability.

\section{Inter-arrival times modeling}
Commonly used probability distributions for modelling inter-arrival times.


\chapter{Main results}
~\cite{MaslovSDM2016, MaslovIJCNN2017}
\section{Pccf}
\section{Integration with Bayesian detector}
\section{Integration with CuSum}

\chapter{MISC}
External~\cite{shewhart1931economic}
Included article~\cite{sha1}.
Concept drift and change detection problem.
Concept drift can be reduced to change detection in univariate time series?


\begin{itemize}
  \item Change detection:~\cite{basseville1993detection}
  \item Sequential change detection problem is a well studied problem, see for example in~\cite{tartakovsky2014sequential}, ~\cite{plasse2021streaming}. 

  \item Optimality of the change detection procedure was investigated in~\cite{Page1954},~\cite{Shiryaev2010,Shiryaev1961,Shiryaev1963}.
  Asymptotic and nonasymptotic optimality of cumulative sum algorithms was provedin~\cite{lorden1971procedures},~\cite{moustakides1986optimal},~\cite{moustakides2004optimality},~\cite{ritov1990decision}. In~\cite{Shiryaev1963,shiryaev2007optimal} the change point is modelled as a random variable with a known geometric distribution~\cite{veeravalli2014quickest} and optimal algorithm minimizing the average detection delay given constraint on the probability of false alarm is proposed. In our work we minimize the detection delay given a constraint on the maximum delay imposed by the prediction interval width. In~\cite{lorden1971procedures} asymptotic optimality of Cusum~\cite{Page1954} is proved according to the minimax criterion for delay with the mean time between false alarms going to infinity.

  \item Concept drift:
\end{itemize}

\chapter{Conclusion}

\tailmatter
\finnishsummary
Foo bar
%\inputencoding{utf8}
\bibliographystyle{plain}
\bibliography{references}
\appendices
\appendix{A}
\section{foobar}

\backmatter

\include{includedarticles, my_articles}
\printindex
\end{document}
