%\usepackage[unicode]{hyperref}
\usepackage{amsmath}								
\usepackage{amsfonts} 							% to make \mathbb work
\usepackage{amssymb}
\usepackage{amsthm}

% graphics
\usepackage{graphicx}								% to make \includegraphics work
\usepackage{subfig}
%
\usepackage{framed}
\usepackage{afterpage}
\usepackage{capt-of}

% tables
\usepackage{multirow}								% multirows in tables
\usepackage{booktabs} 							% correct spacing in tables

% symbols and fonts
\usepackage{dsfont}
\usepackage{bigints} 								% to use big integrals 
\usepackage{xfrac}
\usepackage{mathrsfs}
%\usepackage{mathtools}
\usepackage{textcomp}
\usepackage{upgreek}
\usepackage{units}
\usepackage{bbm}
\usepackage{setspace}
\usepackage{color}
\usepackage{xcolor}


\usepackage{algorithm}              % algorithm environment
\usepackage{algorithmic}

%\usepackage{refcheck}

\usepackage{float}
\floatstyle{plaintop}
\restylefloat{table} % force a table caption on top of the tables
%%--------------------------------------------------------------------------------%%

%\usepackage{tikz}	
%\usetikzlibrary{arrows,chains,matrix,positioning,scopes}
%\makeatletter
%\tikzset{join/.code=\tikzset{after node path={%
%\ifx\tikzchainprevious\pgfutil@empty\else(\tikzchainprevious)% 
%edge[every join]#1(\tikzchaincurrent)\fi}}}
%\makeatother
%\tikzset{>=stealth',every on chain/.append style={join}, every join/.style={->}}
%\tikzstyle{labeled}=[execute at begin node=$\scriptstyle,   execute at end node=$]
%

\usepackage{tikz}

\usetikzlibrary{%
  arrows,%
  shapes.misc,% wg. rounded rectangle
  shapes.arrows,%
  chains,%
  matrix,%
  positioning,% wg. " of "
  scopes,%
  decorations.pathmorphing,% /pgf/decoration/random steps | erste Graphik
  shadows%
}
\tikzset{
  nonterminal/.style={
    % The shape:
    rectangle,
    % The size:
    minimum size=6mm,
    % The border:
    very thick,
    draw=red!50!black!50,         % 50% red and 50% black,
                                  % and that mixed with 50% white
    % The filling:
    top color=white,              % a shading that is white at the top...
    bottom color=red!50!black!20, % and something else at the bottom
    % Font
    font=\itshape
  },
  terminal/.style={
    % The shape:
    rounded rectangle,
    minimum size=6mm,
    % The rest
    very thick,draw=black!50,
    top color=white,bottom color=black!20,
    font=\ttfamily},
  skip loop/.style={to path={-- ++(0,#1) -| (\tikztotarget)}}
}

{
  \tikzset{terminal/.append style={text height=1.5ex,text depth=.25ex}}
  \tikzset{nonterminal/.append style={text height=1.5ex,text depth=.25ex}}
}

%%--------------------------------------------------------------------------------%%
	
\newcommand {\Int}   {\int\limits}
\newcommand {\Sum}   {\sum\limits}
\newcommand {\Sup}   {\sup\limits}
\newcommand {\Max}   {\max\limits}
\newcommand {\Minimum}   {\min\limits}
	
\newcommand {\eps}   {\varepsilon}
\newcommand {\Tau}   {\mathcal{T}}
\newcommand {\F}     {\mathcal{F}}
\newcommand {\Beta}  {\mathcal{B}}
\newcommand {\R}		 {\mathbf{r}}
\newcommand {\I}     {\mathscr{I}}
\newcommand {\Ieff}  {I_{\rm eff}}



\newcommand {\IntQT} {\Int_{Q_T}}
\newcommand {\IntTO} {\Int_{t_k}^{t_{k+1}} \Int_\Omega}
\newcommand {\IntabO} {\Int_{a}^{b} \Int_\Omega}
\newcommand {\IntTGammaN} {\Int_{t_k}^{t_{k+1}} \Int_{\Gamma_N}}
\newcommand {\IntTGammaR} {\Int_{t_k}^{t_{k+1}} \Int_{\Gamma_R}}
\newcommand {\IntT}  {\Int_{t_k}^{t_{k+1}}}
\newcommand {\IntO}  {\Int_\Omega}
\newcommand {\IntGammaN}  {\Int_{\Gamma_N}}
\newcommand {\IntGammaR}  {\Int_{\Gamma_R}}

\newcommand {\waveu} {\widetilde{u}}

\newcommand {\Rd}    {{\mathds{R}}^d}
\newcommand {\Nd}    {{\mathds{N}}}

\newcommand {\Rdd}    {{\mathds{R}}^{d \times d}}
\newcommand {\lstfunc}[1] {\texttt{\small{#1}}}
\newcommand {\constL} {\mathrm{L}}
\newcommand {\operL}{\mathcal{L}}
\newcommand {\PL} {Picard--Lindel\"{o}f\;\;}

\newcommand {\Real}   {{\mathds{R}}}
\newcommand {\Rn}     {{\mathds{R}}^n}
\newcommand {\Rm}     {{\mathds{R}}^m}
\newcommand {\Rone}   {{\mathds{R}}^1}
\newcommand {\Rtwo}   {{\mathds{R}}^2}
\newcommand {\Rthree} {{\mathds{R}}^3}
\newcommand {\Mdd}    {{\mathds{M}}^{d \times d}}

%%--------------------------------------------------------------------------------%%
% norms
\def \NormA#1  {{\mid\!\mid\!\mid #1 \mid\!\mid\!\mid}^2 }   % ||...||
\def \NormAinverse#1  { {\mid\!\mid\!\mid #1 \mid\!\mid\!\mid}^2_* }   % ||...||_*
\def \Normt#1  {\mid\!\mid\!\mid #1 \mid\!\mid\!\mid}   % ||...||
\def \NormQT#1 {{\mid\!\mid\!\mid #1 \mid\!\mid\!\mid}^2_{Q_T}}   % ||...||
\def \NormQO#1 {{\mid\!\mid\!\mid #1 \mid\!\mid\!\mid}^2_{Q^0}}   % ||...||
\def \NormQk#1 {{\mid\!\mid\!\mid #1 \mid\!\mid\!\mid}^2_{Q^k}}   % ||...||
\def \Normf#1  {\Big \lceil #1 \Big\rceil_\Omega}
\def \Normt#1  {\mid\!\mid\!\mid #1 \mid\!\mid\!\mid}   % |||...|||

% operators
\def \dvrg       {\mathrm{div}}	
\def \tr         {\mathrm{tr}}
\def \traspose#1 {{#1}^{rm T}}


% itegrals measures
\def \dt       {\mathrm{\:d}t}
\def \dx       {\mathrm{\:d}x}
\def \dxhat    {\mathrm{\:d}\hat{x}}
\def \dy       {\mathrm{\:d}y}
\def \dxt      {\mathrm{\:d}x\mathrm{d}t}
\def \dst      {\mathrm{\:d}s\mathrm{d}t}
\def \ds       {\mathrm{\:d}s}
\def \dshat    {\mathrm{\:d}\hat{s}}
\def \dl       {\mathrm{\:d}l}
\def \dxi      {\mathrm{\:d}\xi}
\def \d        {\mathrm{\:d}}

% spaces
\def\L#1{L^{#1}}
\def\H#1{H^{#1}}
\def\W#1#2{W^{#1}_{#2}}
\def\C#1{C^{#1}}
\def\Ho#1{H_0^{#1}}
\def\Co#1{C_0^{#1}}
\def\HD#1#2{H^{#1}_{#2}}
\def\tildeH#1{\widetilde{H}^{#1}}
\def\tildeHD#1#2{\widetilde{H}^{#1}_{#2}}
\def\V#1{{V}^{#1}}
\def\VD#1#2{{V}^{#1}_{#2}}

% majorants
\def\M{\overline{\mathrm M}}
\def\Maj{\overline{\mathrm M}^2_{(\delta, \, \gamma, \, \mu)}}
\def\Majmu{\overline{\mathrm M}^2_{(\hat \mu )}}
\def\Majone{\overline{\mathrm M}^2_{(1)}}
\def\Majzero{\overline{\mathrm M}^2_{(0)}}
%
\def\majone{\overline{\mathrm M}^{\,2}_{\mathrm{I}}} % +
\def\incrmajone{\overline{\mathrm M}^{\,2, (k)}_{\mathrm{I}}} % +

\def\maj#1{{\overline{\mathrm M}^2}^{#1}}
%
\def\MajTwo{\overline{\mathrm M}_{(\delta, \gamma, \mu, \epsilon)}}
\def\majtwo{\overline{\mathrm M}^{\,2}_{\mathrm{II}}} % +
\def\majtwo{\overline{\mathrm M}^{\,2}_{\mathrm{II}}} % +
\def\MajII#1{\overline{\mathrm M}_{\mathrm{II}}^{#1}}
%
\def\Min{{\underline{\mathrm M}^2}}
\def\error{{[\,e\,]\,}^2}
\def\mdI{\overline{\mathrm m}^2_{\mathrm{d}}}
\def\incrmdI#1{\overline{\mathrm m}^{2, ({#1})}_{\mathrm{d}}}
\def\mfI{\overline{\mathrm m}^2_{\mathrm{f}}}
\def\incrmfI{\overline{\mathrm m}^{2, (k)}_{\mathrm{f}}}
\def\mfItilde{\tilde{\mathrm m}^2_{\mathrm{f}}}
\def\mfIhat{\hat{\mathrm m}^2_{\mathrm{f}}}
\def\mdItilde{\tilde{\mathrm m}^2_{\mathrm{d}}}
\def\mdIhat{\hat{\mathrm m}^2_{\mathrm{d}}}
\def\majd{{\overline{\mathrm M}^{\,2}_{\,\mathrm{I}, \mathrm{N}}}}

\def\ed{{e}^{\,2}_{\mathrm{d}}}
\def\errorincr{{[\,e\,]\,}^{2, (k)}}
%\def\ed{{\mid\!\mid\!\mid e \mid\!\mid\!\mid}^2}
\def\incred#1{{e}^{\,2, ({#1})}_{\mathrm{d}}}
%\def\incred{{\mid\!\mid\!\mid e \mid\!\mid\!\mid}^{2, (k)}}
\def\et{\overline{\mathrm e}^2_{\mathrm{T}}}

\def\Marker{\mathbb M}
\def\bnorm#1{[\!]\,#1\,[\!]}
\def\Indicator{{{ E}\hskip-5.2pt{ I}}\,}
%%--------------------------------------------------------------------------------%%
\newcommand {\CP}      		{\overline{C}^{\, \mathrm{p}}_{\Gamma}}
\newcommand {\CG}      		{\overline{C}^{\, \mathrm{Tr}}_{\Gamma}}


\newcommand {\CPtetr}     {\widetilde{C}^{\, \mathrm{p}}_{\Gamma}}
\newcommand {\CGtetr}     {\widetilde{C}^{\, \mathrm{Tr}}_{\Gamma}}

\newcommand {\CPThatalphahattheta} {C^{\, \mathrm{p}}_{\widehat{\Gamma}, \hat{\theta}, \hat{\alpha}}}
\newcommand {\CGThatalphahattheta} {C^{\, \mathrm{Tr}}_{\widehat{\Gamma}, \hat{\theta}, \hat{\alpha}}}

\newcommand {\CPThatalphahat} {C^{\, \mathrm{p}}_{\widehat{\Gamma}, \rfrac{\pi}{2}, \hat{\alpha}}}
\newcommand {\CGThatalphahat} {C^{\, \mathrm{Tr}}_{\widehat{\Gamma}, \rfrac{\pi}{2}, \hat{\alpha}}}

\newcommand {\CPTapproxhatalphahat} {C^{\, \mathrm{p}, M}_{\widehat{\Gamma}, \rfrac{\pi}{2}, \hat{\alpha}}}
\newcommand {\CGTapproxhatalphahat} {C^{\, \mathrm{Tr}, M}_{\widehat{\Gamma}, \rfrac{\pi}{2}, \hat{\alpha}}}

\newcommand {\cpalphahat} {c^{\mathrm{p}}_{\rfrac{\pi}{2}, \hat{\alpha}}}
\newcommand {\cgalphahat} {c^{\mathrm{Tr}}_{\rfrac{\pi}{2}, \hat{\alpha}}}


\newcommand {\CPoincare}  {C^{\mathrm P}_{\Omega}}
\newcommand {\CPbound}    {\overline{C}^{\mathrm{P, MR}}_{\Omega}}
\newcommand {\CPTrefpithree}  {C^{\, \mathrm{P}}_{\widehat{\T}, {\rfrac{\pi}{3}}}}
\newcommand {\CPTrefpitwo}  {C^{\, \mathrm{P}}_{\widehat{\T}, {\rfrac{\pi}{2}}}}
\newcommand {\CPTrefpifour}  {C^{\, \mathrm{P}}_{\widehat{\T}, {\rfrac{\pi}{4}}}}

\newcommand {\cpithree}       {\overline{c}_{{\rfrac{\pi}{3}}}}
\newcommand {\cpitwo}       {\overline{c}_{{\rfrac{\pi}{2}}}}
\newcommand {\cpifour}       {\overline{c}_{{\rfrac{\pi}{4}}}}

\newcommand {\mupitwo}      {\mu_{\rfrac{\pi}{2}}}
\newcommand {\mupithree}      {\mu_{\rfrac{\pi}{3}}}
\newcommand {\mupifour}      {\mu_{\rfrac{\pi}{4}}}

\newcommand {\CPTrefleg}  {C^{\, \mathrm{p}}_{\widehat{\Gamma}, {\rfrac{\pi}{2}}}}
\newcommand {\CPTrefhyp}  {C^{\, \mathrm{p}}_{\widehat{\Gamma}, {\rfrac{\pi}{4}}}}
\newcommand {\CGTrefleg}  {C^{\, \mathrm{Tr}}_{\widehat{\Gamma}, {\rfrac{\pi}{2}}}}
\newcommand {\CGTrefhyp}  {C^{\, \mathrm{Tr}}_{\widehat{\Gamma}, {\rfrac{\pi}{4}}}}

\newcommand {\CPTleg}     {\overline{C}^{\, \mathrm{p}}_{{\rfrac{\pi}{2}}}}
\newcommand {\CPThyp}     {\overline{C}^{\, \mathrm{p}}_{{\rfrac{\pi}{4}}}}
\newcommand {\CGTleg}     {\overline{C}^{\, \mathrm{Tr}}_{{\rfrac{\pi}{2}}}}
\newcommand {\CGThyp}     {\overline{C}^{\, \mathrm{Tr}}_{{\rfrac{\pi}{4}}}}
\newcommand {\CPLS}       {\overline{C}^{LS}_{\T}}
\newcommand {\CPlower}    {\underline{C}_{\T}}
\newcommand {\CPupper}    {\overline{C}_{\T}}

\newcommand {\CPTr} {C^{\mathrm{p}}_{\Gamma, r}}
\newcommand {\CGTr} {C^{\mathrm{Tr}}_{\Gamma, r}}

\newcommand {\CPTalpha} {C_{\mathrm{P} \hat{\mathrm{T}}_{\alpha}}}
\newcommand {\CGTalpha} {C_{\Gamma \hat{\mathrm{T}}_{\alpha}}}

\newcommand {\CPTpitwo} {C^{\mathrm{p}}_{\Gamma, {}^{\pi}\!/_{2}}}
\newcommand {\CGTpitwo} {C^{\mathrm{Tr}}_{\Gamma, {}^{\pi}\!/_{2}}}

\newcommand {\CPTpithree} {C^{\mathrm{p}}_{\Gamma, {}^{\pi}\!/_{3}}}
\newcommand {\CGTpithree} {C^{\mathrm{Tr}}_{\Gamma, {}^{\pi}\!/_{3}}}

\newcommand {\CPTpifour} {C^{\mathrm{p}}_{\Gamma, {}^{\pi}\!/_{4}}}
\newcommand {\CGTpifour} {C^{\mathrm{Tr}}_{\Gamma, {}^{\pi}\!/_{4}}}

\newcommand {\CPTtwopithree} {C^{\mathrm{p}}_{\Gamma, {}^{2\pi}\!/_{3}}}
\newcommand {\CGTtwopithree} {C^{\mathrm{Tr}}_{\Gamma, {}^{2\pi}\!/_{3}}}

\newcommand {\CPTref} {C_{\mathrm{P} \widehat{\mathrm{T}}}}
\newcommand {\CGTref} {C_{\Gamma \widehat{\mathrm{T}}}}

\newcommand {\cpleg} {\overline{c}_{\mathrm{p}, {\rfrac{\pi}{2}}}}
\newcommand {\cphyp} {\overline{c}_{\mathrm{p}, {\rfrac{\pi}{4}}}}
\newcommand {\cgleg} {\overline{c}_{\mathrm{Tr}, {\rfrac{\pi}{2}}}}
\newcommand {\cghyp} {\overline{c}_{\mathrm{Tr}, {\rfrac{\pi}{4}}}}

\newcommand {\cpr} {\overline{c}_{\mathrm{p}, r}}
\newcommand {\cgr} {\overline{c}_{\mathrm{Tr}, r}}

\newcommand {\cppitwo} {\overline{c}_{\mathrm{p}, {}^{\pi}\!/_{2}}}
\newcommand {\cgpitwo} {\overline{c}_{\mathrm{Tr}, {}^{\pi}\!/_{2}}}

\newcommand {\cppithree} {\overline{c}_{\mathrm{p}, {}^{\pi}\!/_{3}}}
\newcommand {\cgpithree} {\overline{c}_{\mathrm{Tr}, {}^{\pi}\!/_{3}}}

\newcommand {\cppifour} {\overline{c}_{\mathrm{p}, {}^{\pi}\!/_{4}}}
\newcommand {\cgpifour} {\overline{c}_{\mathrm{Tr}, {}^{\pi}\!/_{4}}}

\newcommand {\cptwopithree} {\overline{c}_{\mathrm{p}, {}^{2\pi}\!/_{3}}}
\newcommand {\cgtwopithree} {\overline{c}_{\mathrm{Tr}, {}^{2\pi}\!/_{3}}}

\newcommand {\approxcpleg} {\underline{c}^{M}_{\mathrm{p}, {\rfrac{pi}{2}}}}
\newcommand {\approxcphyp} {\underline{c}^{M}_{\mathrm{p}, {\rfrac{pi}{4}}}}
\newcommand {\approxcgleg} {\underline{c}^{M}_{\mathrm{Tr}, {\rfrac{pi}{2}}}}
\newcommand {\approxcghyp} {\underline{c}^{M}_{\mathrm{Tr}, {\rfrac{pi}{4}}}}

\newcommand {\Tref} {\widehat{\mathrm{T}}}
\newcommand {\Gref} {\widehat{\Gamma}}
\newcommand {\T} {\mathrm{T}}
\newcommand {\PT} {\mathrm{P}\mathrm{T}}

\newcommand {\approxCPT} {\underline{C}^{M, \mathrm{p}}_{\Gamma}}
\newcommand {\approxCGT} {\underline{C}^{M, \mathrm{t}}_{\Gamma}}
\newcommand {\approxC} 	 {\underline{C}^{M}_{\T}}
\newcommand {\CPT} {C^{\mathrm{p}}_{\Gamma}}
\newcommand {\CGT} {C^{\mathrm{Tr}}_{\Gamma}}


\newcommand {\CFriedrichs} {C_{{\rm F}\Omega}}

\newcommand \Cp[1]  {C_{{\rm P} #1}}
\newcommand \Ctr[1] {C_{{\rm T} #1}}
\newcommand \Ctildetr[1] {\widetilde{C}_{{\rm T} #1}}


\newcommand {\muleg} {\mu_{\rfrac{\pi}{2}}}
\newcommand {\muhyp} {\mu_{\rfrac{\pi}{4}}}

\newcommand {\diam} {\mathrm{diam}}
\newcommand {\meas} {\mathrm{meas}}

\newcommand {\DO} {\mathcal{D} (\Omega)}
\newcommand {\D} {\mathcal{D}}

\newcommand*\rfrac[2]{{}^{#1}\!/_{#2}}
\newcommand{\minus}{\scalebox{0.5}[1.0]{\( - \)}}
\newcommand{\plus}{\scalebox{0.5}[1.0]{\( + \)}}

\def \Mean#1#2 {{ \Big \{ #1 \Big\} }_{#2}}
\def \smallMean#1#2 {{ \big \{  #1 \big\} }_{#2}}

\def\maj#1{\overline{\mathrm M}_{#1}}
\def\majo#1{\overline{\mathrm M}^0_{#1}}
\def\mij#1{\underline{\mathrm M}_{#1}}


\def\majad#1{\overline{\mathrm M}^l_{#1}}
\def\mijad#1{\underline{\mathrm M}^l_{#1}}

\def\majcond#1{\overline{\mathrm M}^{\rm cond}_{#1}}
\def\majauch#1{\overline{\mathrm M}^{\rm A}_{#1}}
\def\mijauch#1{\underline{\mathrm M}^{\rm A}_{#1}}

\def\Pone{{\rm P}_{1}}
\def\Ptwo{{\rm P}_{2}}
\def\RTzero{{\rm RT}_{0}}
\def\RTone{{\rm RT}_{1}}


\newcommand{\mnote}[1]{\!^\textrm{\scriptsize\color{Gray}#1}}
\newcommand{\note}[1]{\textrm{\scriptsize\color{Gray}#1}}
\newcommand{\tnote}[1]{\color{Grayy}#1}


%%--------------------------------------------------------------------------------%%

\newtheorem{theorem}{Theorem}[chapter]
\newtheorem{lemma}{Lemma}[chapter]
\newtheorem{remark}{Remark}[chapter]
\newtheorem{corollary}{Corollary}[chapter]
\newtheorem{definition}{Definition}[chapter]
\newtheorem{proposition}{Proposition}[chapter]
\theoremstyle{definition}
\newtheorem{data}{Data}
\newtheorem{example}{Example}


%\newdefinition{rmk}{Remark}
%\newproof{pf}{Proof}
%\newproof{pot}{Proof of Theorem \ref{thm2}}

%%--------------------------------------------------------------------------------%%


% Default fixed font does not support bold face
\DeclareFixedFont{\ttb}{T1}{txtt}{bx}{n}{12} % for bold
\DeclareFixedFont{\ttm}{T1}{txtt}{m}{n}{12}  % for normal

% Custom colors
\definecolor{Gray}{rgb}{0.6,0.6,0.6}
\definecolor{Grayy}{rgb}{0.5,0.5,0.5}
\definecolor{deepblue}{rgb}{0,0,0.5}
\definecolor{deepred}{rgb}{0.6,0,0}
\definecolor{deepgreen}{rgb}{0,0.5,0}

\usepackage{listings} 

%\DeclareCaptionFont{white}{ \color{white} }
%\DeclareCaptionFormat{listing}{
  %\colorbox[cmyk]{0.43, 0.35, 0.35,0.01 }{
    %\parbox{\textwidth}{\hspace{15pt}#1#2#3}
  %}
%}
%\captionsetup[lstlisting]{ format=listing, labelfont=white, textfont=white, singlelinecheck=false, margin=0pt, font={bf,footnotesize} }


%\lstset{emph={trueIndex,root},emphstyle=\color{BlueViolet}}%\underbar} 
% for special keywords
\lstset{language=[LaTeX]Tex,%C++,
    keywordstyle=\color{blue!50!black},%\bfseries,
    basicstyle=\small\ttfamily,
    %identifierstyle=\color{NavyBlue},
    commentstyle=\color{green!50!black}\ttfamily,
    stringstyle=\rmfamily,
    numbers=none,%left,%
    numberstyle=\scriptsize,%\tiny
    stepnumber=5,
    numbersep=8pt,
    showstringspaces=false,
    breaklines=true,
    frameround=ftff,
    frame=single,
    belowcaptionskip=.75\baselineskip
    %frame=L
} 

% Python style for highlighting
\newcommand\pythonstyle{
\lstset{
		language=Python,
		basicstyle=\footnotesize\ttfamily,
		otherkeywords={self},             % Add keywords here
		keywordstyle=\bfseries\color{deepblue}\bfseries,
		commentstyle=\itshape\color{purple!40!black},
		%emph={MyClass,__init__},          % Custom highlighting
		%emphstyle=\ttb\color{deepred},    % Custom highlighting style
		stringstyle=\color{deepgreen},
		frame=single,                         % Any extra options here
		showstringspaces=false,            % 
		belowcaptionskip=.75\baselineskip
}}
% Python environment
\lstnewenvironment{python}[1][]
{
\pythonstyle
\lstset{#1}
}
{}
% Python for external files
\newcommand\pythonexternal[2][]{{
\pythonstyle
\lstinputlisting[#1]{#2}}}
% Python for inline
\newcommand\pythoninline[1]{{\pythonstyle\lstinline!#1!}}

\setlength{\tabcolsep}{9pt}
\newcommand{\comment}[1]{}

\numberwithin{equation}{chapter}
\def\proof{\noindent\textit{\textbf{Proof}. }}
\def\proofend{\hfill$\square$\vskip+0.5em}

% Definitions from the book
\def\Indicator{{{ E}\hskip-6.0pt{ I}}\,}
\def\Marker{{\mathbb M}}
\def\genmaj#1{\overline{\mathrm M}_{#1}}                % majorant
\def\genmij#1{\underline{\mathrm M}_{#1}}               % minorant
