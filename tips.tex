\documentclass[a4paper,12pt]{article}
\usepackage[T1]{fontenc}
\usepackage[latin1]{inputenc}
\usepackage{url}

\newenvironment{cmds}
  {\begin{description}\renewcommand{\makelabel}[1]{\texttt{\textbackslash##1}\hfill}}
  {\end{description}}

\newenvironment{opts}
  {\begin{description}\renewcommand{\makelabel}[1]{\texttt{##1}\hfill}}
  {\end{description}}

\newenvironment{envs}
  {\begin{description}\renewcommand{\makelabel}[1]{\texttt{##1}\hfill}}
  {\end{description}}

\newcommand{\cmd}[1]{\texttt{\textbackslash#1}}
\newcommand{\opt}[1]{\texttt{#1}}
\newcommand{\env}[1]{\texttt{#1}}
\newcommand{\pack}[1]{\textit{#1}}
\newcommand{\argu}[1]{\textit{#1}}
\newcommand{\cls}[1]{\textsc{#1}}
\newcommand{\prog}[1]{\textbf{\textsf{#1}}}

\author{Matthieu Weber\\ \texttt{mweber@mit.jyu.fi} \and Antti-Juhani Kaijanaho\\ \texttt{antkaij@mit.jyu.fi}}
\title{Tips and Tricks for using \LaTeX\ with \cls{JYdiss}}

\begin{document}
\maketitle


\begin{abstract}
This short document provides some tips and tricks which are not directly
related to the \cls{jydiss} class, but which may be useful for the users of
that class.
\end{abstract}

\tableofcontents

\section{The \pack{hyperref} package}

Hyperref is a very powerful package, but it is also the source of loads of
troubles\dots

One rule to remember is to load \pack{hyperref} \emph{after} all other
packages, unless explained otherwise.

\subsection{\pack{algorithm} and Other \pack{float}-Based Packages}

\pack{algorithm} (and all packages based on \pack{float}) interact badly with
\pack{hyperref}, since they both redefine \cmd{caption}, producing a lot of
warnings when used with \prog{pdflatex}. You must therefore load these packages
\emph{after} loading \pack{hyperref}. The following piece of black magic will
prevent the warnings.

\begin{verbatim}
% patch of float's \caption to avoid double anchor setting by \refstepcounter
% of float's \caption and in hyperref's \@caption. 
 \begingroup
 \makeatletter
 \def\x#1\refstepcounter#2\@nil{%
 \endgroup
 \def\caption{#1\H@refstepcounter#2}%  
}%
\expandafter\x\caption\@nil
\end{verbatim}

\subsection{Colored links in PDF}

By default, when producing PDF output, \pack{hyperref} marks the hyperlinks by
framing the text of the links. This is particularly ugly when the document is
meant to be viewd on a computer monitor (the frames do not appear when the
document is printed). The following command will make the text of the
hyperlink be colored instead of begin framed in color:

\begin{verbatim}
\usepackage[pdftex]{hyperref}
\hypersetup{colorlinks,citecolor=blue}
\end{verbatim}

Note that the \verb"citecolor=blue" is because the default color is IMHO very
ugly on screen, and is completely optional.

There is one drawback, though: when such a document is printed, the colored
text appears in gray on the paper. The best solution is therefore to prepare
two PDF versions of the document, one meant to be printed, with the
\cmd{hypersetup} command from above being commented out (or even without hyperref
at all), and one meant to be used on screen, with the above command activated.

If one chooses to deactivate \pack{hyperref} when producing the printed
version, the \cmd{hypersetup} can be combined with \cmd{usepackage} as
follows:

\begin{verbatim}
\usepackage[pdftex,colorlinks,citecolor=blue]{hyperref}
\end{verbatim}

\section{Centering the item in a \env{figure} or a \env{table}}

Do not use the \env{center} environment in a float (typically \env{figure} or
\env{table}), because \env{center} produces extra vertical space, which is
usually unwanted. Here is a better practice, using the \cmd{centering} command:

\begin{verbatim}
\begin{figure}
\centering
``whatever item goes into the figure''
\caption{My figure}
\end{figure}
\end{verbatim}

\section{Finnish Hyphenation}

If you write your thesis in Finnish, it is useful to have Finnish hyphenation
enabled. You can check if it is enabled by looking for the words
``hyphenation'' and ''finnish'' in the output of \prog{latex}. If it is
similar to the following example, it is already enabled.

\begin{verbatim}
This is e-TeX, Version 3.14159-2.1 (Web2C 7.4.5)
entering extended mode
(./thesis.tex
LaTeX2e <2001/06/01>
Babel <v3.7h> and hyphenation patterns for american, french,
 german, ngerman, finnish, nohyphenation, loaded.
\end{verbatim}

Otherwise, you can enable it by following these instructions (for the
\prog{tetex} distribution in Linux; if you use another distribution, contact
your administrator):

\noindent Edit \url{/etc/texmf/language.dat}, uncomment the line
\smallskip

\verb"%finnish  fi8hyph.tex"
\smallskip

\noindent by removing the \verb"%", and then run
\smallskip

\verb"fmtutil --all"
\smallskip

\noindent if it is in your distribution (it may be a Debian script, I'm not
sure) or if you dont have \prog{fmtutil} go to the directory containing the
tex format files (\url{/var/lib/texmf/web2c} in Debian, otherwise search for a
file called \url{latex.fmt} and go into the directory containing that file)
and run

\begin{verbatim}
tex -ini -jobname=tex -progname=tex tex.ini
tex -ini -jobname=latex -progname=latex latex.ini
etex -ini -jobname=latex -progname=latex *latex.ini
etex -ini -jobname=etex -progname=etex *etex.ini
etex -ini -jobname=elatex -progname=elatex *elatex.ini
pdftex -ini -jobname=pdftex -progname=pdftex pdftex.ini
pdftex -ini -jobname=pdflatex -progname=pdflatex pdflatex.ini
pdfetex -ini -jobname=pdflatex -progname=pdflatex *pdflatex.ini
pdfetex -ini -jobname=pdfetex -progname=pdfetex *pdfetex.ini
pdfetex -ini -jobname=pdfelatex -progname=pdfelatex *pdfelatex.ini
\end{verbatim}

\section{Weird \LaTeX\ Bugs}

\subsection{\cmd{label} and \cmd{caption} in a Float}
When defining a float (figure, table\dots), always put the \cmd{label} command
\emph{just after} the caption (i.e., with no command between the \cmd{caption}
and the \cmd{label} commands). Otherwise, a reference to that label will be
incorrect (it will refer to the current sectioning number instead of the float
number).
\end{document}
