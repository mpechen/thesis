\documentclass[12 pt]{article}
\renewcommand{\baselinestretch}{1}
\usepackage{geometry}
\usepackage{graphics}
\geometry{verbose,letterpaper,tmargin=0.5 cm,bmargin=0.8 cm,lmargin=2.5cm,rmargin=2.5cm,headsep=1cm}
\setlength{\parskip}{\smallskipamount}
\setlength{\parindent}{10 pt}
\usepackage{amssymb}
\usepackage{amsmath}
\usepackage{textcomp}
\usepackage{setspace}
\usepackage{indentfirst}
\usepackage{hyperref}
\title{Predictive analytics with online change detection in data streams}
\date{}
\begin{document}
	\maketitle
  Concept drift (CD) can happen and trained model performance will drop.  In
  order to detect CD change detectors are used to monitor changes in models
  performance metrics on-line. Once change is alarmed the model can be
  re-trained using the relevant data between last change point and current time
  moment.

  Figure XXX illustrates CD phenomena SINE1. 
  Figure XXX illustrates SVM decision boundary before and after CD in SINE 1 dataset.
  Performance drops after moment $t$. 
  And detectors alarms the change at the moment $t_1$.

  Figure XXX illustrates an example of adaptive learning with detector and
  without. It can be seen that with detector performance is better and with
  recurrent detector it is even more better because we can skip possible FA and
  we can possibly reduce the detection delay.

\end{document}
