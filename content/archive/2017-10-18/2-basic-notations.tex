\chapter{Basic definitions and notations}
%---------------------------- SOURCES ----------------------------------------%
% 1. All of Statistics
% 2. PGM Coursera lectures
%
% http://www.cs.ucr.edu/~eamonn/online_motifs.pdf:
% Time series definition is in "Online Discovery and Maintenance of Time Series Motifs" by Abdullah Mueen, Eamonn Keogh
%
% Box, Jenkins 2.1:
% "A time series is a set of observations generated sequentially in time. If the set is continuous, the time series is said to be continuous. If the set is discrete, the time series is said to be discrete"
%
% By Tommi: DOBRObasicdefs.tex
%--------------------------END SOURCES ---------------------------------------%
\noindent
\textbf{- Time series definition}\\

Good definition for time series / input signal is in~\cite{gilbert2001surfing}.

the signal a is a one dimensional function:$a: [0,\dots,(N-1)] -> Z^+$

\begin{definition}
    %% why ordered?
    % is an ordered
    A univariate time-series $\mathbf{y}$ is a sequence of indexed values
    \[
    \mathbf{y} 
    \equiv \lr{y (t_1), \ldots, y (t_T)} 
    \equiv \lr{y (t_i)}_{i=1}^T 
    \equiv \lr{y_1, \ldots, y_T} 
    \equiv \lr{y_i}_{i=1}^T
    \]
    where $t_i$ is the sampling time of the $i$'th observation $y_i = y (t_i)$; 
\end{definition}
$T$ determines the number of points, that is, the length of the time series; 
%% Why an ordered?
% and $\langle \cdot \rangle$ denotes an ordered row vector. And where
and $\lr{\cdot}$ denotes a column vector. 
And where
\begin{itemize}
%\item $\langle \cdot \rangle$ - ordered sequence / vector of elements
\item $\lr{t}_{t=1}^T \equiv \lr{1, 2, \dots, T}$ - vector of moments of time when observations were taken
\item $y(t_i)$ - observation taken at the moment $t_i \in \lr{t}$
\end{itemize}

%% From Dobro poster version
Let $ \lr{y_t} \equiv \lr{y_1,\dots, y_T}$ be the series of observed at time moments $t \in \lr{1,\dots, T}$ values of the response variable $\mathbf{y}$, e.g. bus arrival time.
Denote $\lr{\hat{y}_t}$ as a series of black-box model's predictions made with prediction horizon $h$.
Denote $\lr{r_t}$ as the residuals, i.e. $r_t = \hat{y}_t - y_t$.
% 23.1 Wasserman
%$ \langle y_t : t \in \langle 1,\dots, T \rangle \rangle$\\

In this work we consider time series data generated by sensors.
The typical task is on-line change detection.
This is an important practical task in many areas, e.g. process control and monitoring (bridges, fault prevention).

\noindent
\textbf{- Definitions of change points; TP/FPs; Reurrency and periodicity}\\
Following~\cite{mackay2007} we introduce notation of the current `run length'.
%\begin{definition}
%    Run length $r_t$ is a time since the last change point.
%    \label{def:run_length}
%\end{definition}
\begin{definition}
	Change point is a moment of time when statistical properties of the data stream change significantly according to the predefined criteria.
    Run length $r_t$ is a time since the last change point.
    % \langle c_i \rangle_{i=1}^k = \{ \langle c_1, c_2, \dots, c_k \rangle \in \langle t \rangle_{t=1}^T \: | \: c_1 < c_2 \cdots < c_k \}
    %\[ \langle c_i \rangle_{i=1}^k \in \langle t \rangle_{t=1}^T \]
    At change points run length drops to zero ($r_t \equiv 0 $).
    \begin{equation}
        \lr{c_i}_{i=1}^k \in \lr{t}_{t=1}^T
    \end{equation}
    \[
        \{ c_i,  | \}
    \]
\end{definition}
\begin{definition}
    Change detection event (CDE) is a moment of time when detector alarmed change.
    Vector of change detection events (CDEs) is denoted as
    \[
    \lr{c^\ast_i}_{i=1}^{l} \in \lr{t}_{t=1}^T
    \]
\end{definition}
Changes in sensor data reflect changes in the underlying physical process (phenomenon) being observed.
Change takes some time like any physical process.

\begin{figure}
%    \centering
    \begin{tikzpicture}[scale = 1]
    %\draw [<->] (0, 1.2) node [left] {$p$} -- (0,0) -- (4.0, 0) node [below right] {$\tau$};
    % Y and X axes
    \draw [->] (0.0 , -0.5) -- (0, 5.0);
    \draw [->] (-0.5, 0.0) -- (14.0, 0.0); 
    % Axis labels
    \node [below] at (14.0, 0.0) {$T$};
    \node [left]  at (0.0, 5.0) {$y(t)$};
    \end{tikzpicture}
\end{figure}  

- we introduce expected time duration of the change process (related to the detection delay by the detector)
- change detection regions 
- overlap of CDEs and change detection regions are TPs
- to reference the probability distribution of segment z we add a segment index subscript, such as $G_z$, to denoted Pdf at time within this segment.
%There are gradual, abrupt changes etc.
We consider changes actually as single moments defined by the time when some particular change detector alarms the change.

\begin{definition}
    The set of TPs is an intersection of the set of change points and the set of 
\end{definition}
TP / FP / TN / FN definitions.\\
%... USE https://en.wikipedia.org/wiki/Jaccard_index ?
TP is any $c_j^\ast, j \in \lr{ 1,\dots,l }$ such that 
\[
\exists \: i \in \lr{ 1,\dots, k } : |c_i - c_j^\ast| \leq d
\]
FP is any $c_j^\ast, j \in \lr{ 1,\dots,l }$ such that
\[
\nexists \: i \in \lr{ 1,\dots,k } : |c_i - c_j^\ast| \leq d
\]
TN is any $t \in \lr{ 1,\dots,T }$ such that (!!! add $d$ neighborhood for operation of union of the sets)
\[
\forall i \in \lr{ 1,\dots,k }, \forall j \in \lr{ 1,\dots,l}: c_i \neq t , c_j^\ast \neq t 
\]
FN is any $c_i, i \in \lr{ 1,\dots,k }$ such that  
\[
\nexists \: j \in \lr{ 1,\dots,l } : |c_i - c_j^\ast| \leq d
\]

\begin{definition}
Changes $\lr{c_i}$ are recurrent if 
\end{definition}
\begin{definition}
Changes $\lr{c_i}$ are periodic if 
\end{definition}
%
%%%%% PROPOSED BY TOMMI
%\subsection{DOBRO notations}
%%%.... WHY? DEFINITIONS FROM THE PAPER ARE OK?
%Further definitions:
%\begin{itemize}
%	\item Let $b_i = b (t_i)$ be the value of the black-box prediction of $y_i$, and denote by $e_i = y_i - b_i$ the $i$'th prediction error~\footnote{Residual is predicted value minus actual value $e_i = b_i - y_i$}
%	%(THINK: $r$ or $e$; what is the difference if we refer this/these as errors versus as residuals?)
%	\begin{itemize}
%		\item State here the conditions that we assume on $\langle \mathbf{e} \rangle$
%	\end{itemize}
%	\item Define $c_i = c (t_i)$ be the value of a corrector that tries to decrease the $i$th error $e_i$.
%	\begin{itemize}
%		\item State here the conditions that we assume on $\langle \mathbf{c} \rangle$
%	\end{itemize}
%\end{itemize}
%% ... END from DOBRObasicdefs.tex:
%%.... END WHY? DEFINITIONS FROM THE PAPER ARE OK?
%%\newpage
%%%%% END PROPOSED BY TOMMI

\noindent
- \textbf{Probability notations}\\
Probability density is \\
%%%... Gaussian distribution
Gaussian distribution is given by Equation~\ref{eq:gaussian_distribution}
\begin{equation}
\GaussDist
%\mathcal{N}(x | \mu, \sigma)  =\frac{1}{(2 \pi \sigma^2)^{1/2}} \exp \Big (- \frac{(x-\mu)^2}{2 \sigma^2} \Big )
\label{eq:gaussian_distribution}
\end{equation}
%%%... End Gaussian distribution

From~\cite{MacKay_Inference_Book}
\begin{definition}
    An ensemble $X$ is a triple $(x, \mathit{A}_X, P_X)$, where outcome $x$ is the value of a random variable, which takes on one of the set of possible values, $\mathit{A}_X = \{a_i\}_{i=1}^I$ having probabilities $\mathit{P}_X = \{p_k\}_{k=1}^I$ with $P(x=a_i)=p_i, p_i \geq 0$ and $\sum_{a_i \in \mathit{A}_X} P(x=a_i) = 1$.
    \label{def:ensemble}
\end{definition}

\begin{definition}
    Markov assumption / Markovian stochastic process
    \begin{equation}
    (X^{(t+1)} \perp X^{(0:t-1)} \: | \: X^{(t)})
    \label{eq:markov_assumption}
    \end{equation}
\end{definition}

%%%%.... TABLE OF MATH NOTATIONS
%\newpage
%Notations are summarized in the table~\ref{tab:notations_summary}.
%\begin{table}[htb!]
%    \centering 
%    %    \begin{tabular}{|l  l|}\hline
%    \begin{tabular*}{\textwidth}{l l}
%        \hline \\
%        Row vector                            & $\langle \cdot \rangle$                \\
%        Time moments                          & $\langle t \rangle_{t=1}^T$            \\ 
%        Observation taken at the moment $t_i$ & $y(t_i), \: t_i \in \langle t \rangle$ \\
%        Time series                           & $\langle y_i \rangle_{i=1}^T$          \\ 
%        Independence                          & $X^{i+1} \perp X^i$                    \\
%        Set of states / items                 & $\{S_i\}_{i=1}^N$                      \\
%        %\hline \\
%        Probability of event                  & $P(x=a_i)$                             \\
%        \hline
%    \end{tabular*}
%    \caption{Notations summary}
%    \label{tab:notations_summary}
%\end{table}
%%%%.... END TABLE OF MATH NOTATIONS
%%%%.... TABLE OF ABBREVIATIONS
%\begin{table}[htb!]
%    \centering 
%    \begin{tabular*}{\textwidth}{l l}\hline
%        True Positive / Negative   & TP / TN \\
%        False Positive / Negative & FP / FN \\
%        Change Detection Event         & CDT \\  \hline
%    \end{tabular*}
%    \caption{Abbreviations}
%    \label{tab:abbrev_summary}
%\end{table}
%%%%.... END TABLE OF ABBREVIATIONS
