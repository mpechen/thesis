\chapter{Conclusions and outlook}
\label{chapter:conclusion}

In this chapter, we summarize the results presented in the study and give an outlook 
on some future research.
%--------------------------------------------------------------------------------------%
The first part of the thesis presents a new version of the \PL method for nonlinear 
ODEs supplied with guaranteed and explicitly computable upper bounds of approximation 
errors. The estimates derived take into account interpolation and integration errors 
and, therefore, provide objective on the accuracy of computed approximations 
(see \cite{RefMatculevichNeittaanmakiRepin2013}). 

In the second (major) part of the work, guaranteed bounds of distance to the exact 
solution of the evolutionary reaction-diffusion problem with mixed BC are discussed. We 
show that two-sided error estimates are directly computable and equivalent to the error. 
Numerical experiments have confirmed that the estimates provide accurate two-sided 
bounds of the overall error and generate efficient indicators of local error 
distribution (see \cite{RefMatculevichRepin2014} and Section \ref{sec:numerical-example} 
of the current work). 

Earlier, we have generalized two-sided bounds to evolutionary reaction-diffusion 
problems and adapted them to domains of complicated structure with mixed 
Dirichlet--Robin BC. The estimates are also valid for problems with complicated 
nonlinear source functions. To overcome computational difficulties, the domain 
decomposition method was used. To obtain the error estimate, we have exploited the 
classical Poincar\'{e} and Poincar\'{e}-type inequalities for functions with zero mean 
boundary traces. Therefore, the new corresponding bounds of the distance to the exact 
solution contain only constants in local Poincar\'e-type inequalities associated with 
subdomains, which quantitatively improves the majorant value. Moreover, it has been 
proved that the bounds are equivalent to the primal and primal-dual energy norms of 
the error (see 
\cite{RefMatculevichNeitaanmakiRepin2015, RefMatculevichRepinPoincare2014}).

%--------------------------------------------------------------------------------------%
The above-introduced estimates require exact values of guaranteed and realistic bounds 
of constants in respective functional inequalities. Therefore, in the last part of the 
thesis, we present sharp estimates of constants in Poincar\'{e} and 
Poincar\'{e}-type inequalities for functions having zero mean value on the boundary of 
a Lipschitz domain or on a measurable part of it. These estimates are particularly used 
in a posteriori error estimation methods for I-BVPs introduced in 
\cite{RefMatculevichNeitaanmakiRepin2015} and \cite{RefMatculevichRepinPoincare2014}. 
Our focus was on computable relations that provide sharp bounds of the constants in the 
above-mentioned inequalities on simplexes in 2D and 3D, which, based on numerical 
simulations, have confirmed to provide efficient numerical results. Also, we have 
numerically studied the behavior of the constants in the classical  Poincar\'{e} 
inequalities and compared these results with known analytical estimates.

In the context of partial differential equations, we have studied only linear models. 
Thus, extension of these methods to nonlinear I-BVPs is a matter of future work. 
Moreover, it would be interesting to extend the application of majorant for 
nonconforming approximations. The estimates based on the domain decomposition technique 
and local classical Poincar\'{e} and Poincar\'{e}-type inequalities for functions with 
zero mean trace obtained in \cite{RefMatculevichRepinPoincare2014} can be further 
developed. Finally, one of the most important directions for the future work is to 
improve the speed of majorant reconstruction, e.g., to implement a highly parallel 
algorithm for its minimization. 

