\documentclass[paper=a4, fontsize=12pt]{article}
%\usepackage[T1]{fontenc}
%\usepackage{fourier}
\usepackage[english]{babel}															% 
\usepackage{amsmath,amsfonts,amsthm} % Math packages
\usepackage[pdftex]{graphicx}	
\usepackage{url}
\usepackage{hyperref}
\title{Draft}
%\vspace{-1in} 	
%\date{}
\begin{document}
%\boldmath
\maketitle
\section{Overall framework}
Consists of three parts
\begin{enumerate}
	\item Recurrency
	\item Context (through recurrency)
	\item Multi objects
\end{enumerate}
%Book~\cite{Box_Jenkins_Arima}

\section{Multi sensor}
Building blocks are in papers ~\cite{MacKay2007},~\cite{Chapados2014},~\cite{Rasmussen2010} ,~\cite{MacKay_Inference_Book}

An idea is to generalize an approach developed in~\cite{MacKay2007} for multi-sensor settings.

%%%%%%%%%%%% DOBRO

\section{Dobro Arima}

Correction principle 
\begin{equation}
P(\underline{\hspace{12px}} | (\mu, \sigma,) \lambda
\end{equation}
Probability $P(\mu,\sigma)$ where $(\mu,\sigma)$ are random shocks parameters in the infinite representation of ARIMA gives us confidence intervals.
Confidence intervals(CI) give probability of a sign change.
At the same time probability of a sign change is also related to the expected time interval between consecutive sign changes (periodicity).

Inputs for Dobro are 1) vector of the black-box predictions up to current moment of time $\hat Y_t$, 2) vector of known residuals $r_{t-l}$ (or equivalently - true observations $Y_{t-l}$).
% Two cases : residuals are stationary and not.

About correlations in forecasts~\footnote{Important (~\cite{Box_Jenkins_Arima} p.141): correlation between the forecasts errors. "..the optimal forecasts errors at lead time 1 will be uncorrelated, the forecast errors for longer lead times in general will be correlated."}

Components of the Dobro are
\begin{enumerate}
\item Correction principle.
\item Error forecasts taking into account correction principle and assuming that sign is correct / probability of the sign change.
\item Probability of the sign change of residuals made by the black box model.
\item Bayesian ARIMA model of a special form where length of the vector of the black-box predictions $Y_t$ is greater than length of the vector of residuals $r_{t-l}$?
\item Contextual variable affecting joint probability distribution?
\end{enumerate}

\subsection{When correction fails?}
Given probability distribution of residuals $P_r$ correction term is the conditional expectation of $y_{t+l}$ at current time $t$: $E(\hat y_{t+l} | y_t, y_{t-1}, \dots )$.
Correction procedure fails if concept drift happens in the future and underlying distribution $P_r$ changes. Concept drift itself might be predictable or unpredictable.
 
\textbf{Basic models (MA, Naive, Exp smoothing) in terms of Arima}.\\
Accordingly to~\cite{Box_Jenkins_Arima} (p.11, 100) ARIMA model in general case can be written as 
a) for stationary process
\begin{equation}
\phi(B) \nabla^d \tilde{z_t} = \theta(B) a_t
\label{eq:arima_stat}
\end{equation}
where $\tilde{z_t} = z_t - \mu$.
b). For non stationary case as
\begin{equation}
\phi(B) \nabla^d z_t = \theta(B) a_t
\label{eq:arima_not_stat}
\end{equation}
Note~\footnote{$(1-B)^d  == \nabla^d$} that
(\cite{Box_Jenkins_Arima}, p.12) if we replace $\nabla^d z_t$ by $z_t - \mu$ when $d=0$ , the model~\ref{eq:arima_not_stat} includes the stationary model~\ref{eq:arima_stat} as a special case.
 
\begin{equation}
\phi(B) = 1 - \phi_1 B - \phi_2 B^2 - \dots - \phi_p B^p
\label{eq:phi_b}
\end{equation}

\begin{equation}
\theta (B) = 1 - \theta_1 B - \theta_2 B^2 - \dots - \theta_q B^q
\end{equation}

In our case residuals (errors) are available with some time delay  $l$. For an analysis available only residuals from the time moment $t - l$ and therefore we specify model as 
\begin{equation}
\phi(B) (1-B)^d z_t = \theta^l(B) a_t
\end{equation}
where
\begin{equation}
\theta^l (B) = - \theta_{t-l} B^{t-l} - \dots - \theta_q B^q
\end{equation}

Also we introduce probability of the concept drift / sign change of the residuals from the black-box model.

DOBRO is a function plus probability of correction
\begin{align}
&f(\phi (B), \theta^l (B), l, Y_t)\\
&P_{corr}(\lambda)
\label{eq:dobro}
\end{align}
plus
Condition of successful correction

\href{https://stats.stackexchange.com/questions/23864/what-common-forecasting-models-can-be-seen-as-special-cases-of-arima-models}{stats.stackexchange.com},~\cite{Nau}.
\begin{enumerate}
\item ARIMA(0,0,0) = Mean / Moving average
\item ARIMA(0,1,0) = Naive  Random walk (non stationary) 
\item ARIMA(0,1,1) = Simple exponential smoothing
\item ARIMA(1,) = Simple regression 
\end{enumerate}

\subsection{Correction from ARIMA poi t of view}
p.148 (5.2.3)
\begin{equation}
\hat{z}_{t+1} (l) = \hat{z_t}(l+1) + \psi_l a_t
\end{equation}

\subsection{Probability of the sign change}


\section{Notes}
\subsection{Forecast intervals for ARIMA(1,1,1) process}
% http://stats.stackexchange.com/questions/167929/coefficients-in-the-random-shock-form-of-arima-model
ARIMA(1,1,1) process can be written as Eq.~\ref{eq:arima_111}, ~\cite{Box_Jenkins_Arima} (p.102, 4.1.3).
%\begin{equation}
%\nabla z_t - \phi_1 \nabla z_{t-1} = a_t - \theta_1 a_{t-1}
%\end{equation}
\begin{equation}\label{eq:arima_111}
(1 - \phi_1 B) \nabla z_t = (1 - \theta_1 B) a_t
\end{equation}
where $\nabla = (1-B)$. To calculate confidence intervals (CI) we should consider representation (4.2.2, p.104, 5.1, p.138) as infinite weighted sum of current and previous shocks (random shock form of the model) (Eq.~\ref{eq:randomshockmodel})
\footnote{Note that random shocks are calculated as a difference between actual and predicted values: $a_t = z_t - \hat z_{t-1}(1)$, (5.2, page 147).}
\begin{equation} \label{eq:randomshockmodel}
z_{t} = a_t + \sum_{j=1}^{\infty} \psi_j a_{t+l-j} = \psi(B) a_t
\end{equation}
where $\psi_0=1$ and weights may be obtained by equating
\begin{equation} \label{eq:psi_weights_general}
\phi(B)(1 + \psi_1 B + \psi_2 B^2 + \dots ) = \theta (B)
\end{equation}
For ARIMA(1,1,1) process $\phi(B) = (1-\psi b)(1-B) = 1 - (1+\psi)B + \psi B^2$ and $\theta(B) = 1 - \theta B$. Eq.~\ref{eq:psi_weights_general} will take a form

\subsection{Bayesian hierarchical ARIMA models}
We have a set of correction models.

\subsection{Bayesian Arima}
\newpage`
\bibliographystyle{plain}
\bibliography{Refs}
\end{document}