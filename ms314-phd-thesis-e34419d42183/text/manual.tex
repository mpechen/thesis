\documentclass[a4paper,12pt]{article}
\usepackage[T1]{fontenc}
\usepackage[latin1]{inputenc}
\usepackage{url}

\newenvironment{cmds}
  {\begin{description}\renewcommand{\makelabel}[1]{\texttt{\textbackslash##1}\hfill}}
  {\end{description}}

\newenvironment{opts}
  {\begin{description}\renewcommand{\makelabel}[1]{\texttt{##1}\hfill}}
  {\end{description}}

\newenvironment{envs}
  {\begin{description}\renewcommand{\makelabel}[1]{\texttt{##1}\hfill}}
  {\end{description}}

\newcommand{\cmd}[1]{\texttt{\textbackslash#1}}
\newcommand{\opt}[1]{\texttt{#1}}
\newcommand{\env}[1]{\texttt{#1}}
\newcommand{\bibfld}[1]{\texttt{#1}}
\newcommand{\bibstr}[1]{\texttt{#1}}
\newcommand{\pack}[1]{\textit{#1}}
\newcommand{\argu}[1]{\textit{#1}}
\newcommand{\cls}[1]{\textsc{#1}}
\newcommand{\prog}[1]{\textbf{\textsf{#1}}}

\author{Matthieu Weber\\ \texttt{mweber@mit.jyu.fi} \and Antti-Juhani Kaijanaho\\ \texttt{antkaij@mit.jyu.fi}}
\title{\cls{JYdiss} --- A Class for Writing Doctoral Dissertations at the Faculty of
Information Technology of the University of Jyv�skyl�}

\begin{document}
\maketitle

\begin{abstract}
This class is designed to facilitate the writing of a doctoral dissertation
(or a Licentiate's thesis) at the Department of Information Technology of the
University of Jyv�skyl�. It may also be used elsewhere if it conforms to
their requirements.
\end{abstract}

\section{Read this first}

The class is provided with the hope that it is useful but no assurance
of correctness.  We urge you to consider it your honor-bound duty to
\emph{let us know of any deficiencies} that you may find when using
this class.  First, however, please verify that any strangeness you
may witness is not mandated by the university publishing unit
specifications!

While this class has already been used to typeset several subsequently
approved theses successfully, \emph{it is always your responsibility,
  as the author,} to ensure that the document meets the requirements
set by the university and the faculty in question.  This class aids in
that, but it does not, and cannot, implement the all of the
typesetting requirements.

Please, subscribe to the mailing list Tutkielma-TeX
(\url{http://lists.jyu.fi/mailman/listinfo/tutkielma-tex}), and ask
there any questions you might have.  The subscribers there may be able
to help you faster than we can.

We make changes to the class from time to time.  Class development
happens and releases are provided in Yousource (see
\url{https://yousource.it.jyu.fi/latex-thesis-classes})
We will announce new releases on the mailing list.

\section{Invocation}

Invoke the class with the usual \cmd{documentclass\{jydiss\}}. In order to
write a Licentiate's thesis, use the \opt{licentiate} option.

\subsection{Input Character Set Options}

Input character set options allow to specify with what character set the
source file is written. Available character sets are:

\begin{opts}
\item[ascii] ASCII encoding (ISO 646).
\item[utf8] UTF-8 (Unicode) (requires recent inputenc package).
\item[utf8x] UTF-8 (Unicode) using ucs.sty.  Incompatible with biblatex.
\item[latin1] ISO-8859-1 encoding (Western Europe languages) (default).
\item[latin2] ISO-8859-2 encoding (Central Europe languages).
\item[latin3] ISO-8859-3 encoding (Esperanto, Maltese).
\item[latin5] ISO-8859-5 encoding (Cyrillic).
\item[latin9] ISO-8859-15 encoding (Western European languages with the Euro
symbol).
\item[applemac] Old Macintosh encoding.
\item[ansinew] Windows 3.11 ANSI (extended ISO-8859-1) encoding.
\item[cp1252] synonym for \opt{ansinew}.
\item[cp1250] MS Windows 1250 (central and eastern Europe languages) code
page.
\item[decmulti] DEC Multinational Character Set encoding.
\item[next] Next encoding.
\item[cp437] IBM 437 code page.
\item[cp437de] IBM 437 code page (German version).
\item[cp850] IBM 850 code page.
\item[cp852] IBM 852 code page.
\item[cp865] IBM 865 code page.
\end{opts}

Usual encodings, by platform, are:

\begin{description}
\item[Linux] \texttt{latin1} (or nowadays more often than not, \texttt{latin9}
or \texttt{utf8}).
\item[MS Windows] \texttt{cp1252}.
\item[Old Macintosh] \texttt{applemac}.
\item[OS X] \texttt{utf8}.
\end{description}

\subsection{Package-related Options}

\begin{opts}
\item[index] will add support for an index in the document, based on the
method using the \prog{makeindex} external program. It loads the
\pack{makeidx} package, inserts the index's title in the body of the thesis
and into the table of contents. Actually including the index is done by
putting \cmd{printindex} at the desired place in the document.
\item[subfigure] will tell the class that the \pack{subfigure} package
is being used (incompatibilities between \pack{tocloft} and \pack{subfigure}
make it necessary to tell \pack{tocloft} that \pack{subfigure} is being used).
\item[listings] will load and configure \pack{listings} to conform to
  the University requirements.
\end{opts}

\subsection{General Layout Options}

\begin{opts}
\item[licentiate] will change some layout details to match the Licentiate's
theses specs.
\item[finnish] will enable support for Finnish language (probably incomplete).
Note that this option can be used only if the licentiate option has been
selected since doctoral theses in IT are always in English (this restriction
can be removed if needed).
\item[lof] will include the list of figures into the document.
\item[lot] will include the list of tables into the document.
\item[loa] will include the list of algorithms into the document (to be used
with the \pack{algorithm} package).
\item[loar] will include the list of included articles into the document.
\item[shortloft] will put the lists of figures and tables on the same page, if
they are short. \emph{Do not use it if you do not have a list of figures}.
\item[contribinloar] will put the content of the \env{contribution}
environment in the list of included articles. If this option is not set, the
content of \env{contribution} will be ignored, and the author is free to put
the contribution's text anywhere in the document.
\item[contribbefore] will place the description of the author's contribution
before the list of included articles instead of after. The \opt{contribinloar}
option must be set for this one to have any effect.
\item[bibweaklang] has effect only if the \emph{jydiss} bibliography
  style is used with Bib\TeX; it will restrict the effect of any
  language field in a Bib\TeX{} record to hyphenation only.  Without this
  option, using the \emph{jydiss} bibliography style, the language
  field will affect also the overall language used in the entry
  (things like ``in'' versus ``teoksessa'' etc.).
\end{opts}

\subsection{Layout Fine-Tuning Options}

\begin{opts}
%\item[captiondot] will add a period at the end of the captions (figures and
%tables).
\item[altlongcaption] will break long captions (i.e. captions which are longer
than the width of the text) into a paragraph which is aligned to the left
margin instead of being justified to the right of the label.
\item[alttt] will use the \emph{TXTT} typewritefont instead of \emph{Courier}.
According to some, \emph{TXTT} is looking better than \emph{Courier} when
typset along \emph{Palatino}.
\item[boldartref] will set the in-text references to the included articles in
bold instead of being surrounded by square brackets.
\end{opts}

\section{Preamble Commands}

The following commands can be used before the \cmd{begin\{document\}}. Some
of those are optional (their default values are described along with the
command) and the others are mandatory. If one of the mandatory commands is not
used, a reminder will be printed inside the document.

\begin{cmds}
\item[title] document's title (mandatory).
\item[subtitle] document's subtitle (optional).
\item[entitle] document's title in English (mandatory in Finnish documents, ignored otherwise)
\item[setauthor] document's author (two arguments: first names and surname)
(mandatory).
%\item[setdate] takes three arguments: day, month, year (all numeric);
%  sets document's publishing date (defaults to the current date)
\item[disstype] type of work (defaults to ``Dissertation draft manuscript'').
\item[abstract] abstract in English (mandatory).
\item[keywords] document's keywords in English (mandatory).
\item[people] list of the people involved in the work (see also below).
\item[epigraph] a quotation, dedication or other similar note (not to
  be confused with the acknowledgements section) set on a page of its
  own, vertically centered, just before the abstract
\item[email] typesets an e-mail address (see also below).
\item[isbn] set the ISBN of the thesis (to be obtained from the
  library when the thesis is ready to be published) (mandatory;
  multiple uses allowed; see also below).
\item[issn] set the ISSN of the thesis (not needed for IT faculty
  theses published in the University series)
\item[series] set the name of the series in which the thesis will be
  published (not needed for IT faculty theses published in the
  University series)
\item[serialnumber] set the thesis' serial number (to be obtained from
  the library when it is ready to be published).
\end{cmds}

The \cmd{people} command is used differently from the other ones, since it
contains \cmd{item} commands, as a list would do. The syntax for the
\cmd{item} is as follows:
\verb,\item[,\argu{role}\verb,]{,\argu{information}\verb,},
where \argu{role} is the role of the person (e.g. author, supervisor, opponent,
reviewer, examiner), and \argu{information} is the contact information for that person
(e.g. name, organization,$\,\cdots$)

The \cmd{isbn} command takes an optional argument that, when present
and nonempty, is typeset after the ISBN itself in parentheses.  The
command can also be repeated, to specify more than one ISBN.  For
example
\begin{verbatim}
\isbn[nid.]{123-456-78-9012-3}
\isbn[PDF]{345-678-90-1234-5}
\end{verbatim}
produces
\begin{quote}
ISBN~123-456-78-9012-3~(nid.)\\
ISBN~345-678-90-1234-5~(PDF)
\end{quote}
in the appropriate place on the abstract page.  Please note that these
ISBNs are fictitious and must not be used in real theses.

The \cmd{email} command can be used anywhere in the document, but more
particularily in \cmd{people}, for including the author's e-mail address.

\section{Sectioning Commands}

The available sectioning commands are:
\begin{cmds}
\item[preface] Preface.
\item[acknowledgements] Acknowledgements.
\item[termlist] Glossary.
\item[mainmatter] Marks the beginning of the main part of the document.
Should appear before the first \cmd{chapter}. It mainly includes all the
``List of'' (if any), the table of contents, and the list of articles (if any).
\item[tailmatter] Marks the end of the body of the document. Should
appear before the bibliography and before the Finnish Summary (Yhteenveto) for
proper page numbering in these chapters.
\item[backmatter] Marks the end of the main part of the document. Should
appear after \cmd{appendice} (if any) and before \cmd{includedarticles}.
Chapters are not allowed anymore after \cmd{backmatter}.
\item[chapter] Beginning of a chapter.
\item[section] Beginning of a section.
\item[subsection] Beginning of a subsection.
\item[subsubsection] Beginning of a subsubsection.  (NOTE: Not
  specified by the University Library guidelines.  Use at your own
  risk.)
\item[bibliography] allows to specify a list of references. Should be put
before the appendices.
\item[appendices] Marks the beginning of the appendices. The \cmd{chapter}
command should not be used anymore, use \cmd{appendix} instead. It also
changes the behavior of \cmd{section} and \cmd{subsection} so that the word
``Appendix'' is prepended to it.
\item[appendix] Like \cmd{chapter}, but prepends the word ``Appendix'' in
front of the number.
\item[includedarticles] inserts an ``Included Articles'' line to the table of
contents. The \env{article} environment can be used below it to include
articles.
\item[finnishsummary] begins the ``Yhteenveto (Finnish Summary)''
  chapter and adds the ``Finnish summary'' entry in the abstract page.
  See also the \env{yhteenveto} environment below.
\item[printindex] includes the index in the document (see also the \opt{index}
option.
\end{cmds}

Note that \cmd{subsubsection}, \cmd{paragraph} and \cmd{subparagraph} are
available, but not recommended.
\bigskip

In addition to the previous commands, the following environments is available:

\begin{envs}
\item[yhteenveto] Useful for writing the Finnish summary of an
  English-language thesis.  The environment uses \cmd{finnishsummary}
  to typeset the summary chapter's title, and selects Finnish as the
  language for the duration of the environment.  Under the Publishing
  Unit's rules, the yhteenveto belongs near the end of the thesis,
  before the bibliography.
\item[acronyms] Adds a chapter called ``Acronyms'' (the name can be changed by
using the \cmd{setacronyms} command). Each individual acronym is specified
using the \cmd{item} command, with the following syntax:
\verb,\item[,\argu{acronym}\verb,]{,\argu{full text}\verb,}, where \argu{acronym}
is the given acronym, and \argu{full text} is it meaning.
\item[article] Adds a title page (right-hand) for an included article (which
is inserted to the document after printing. The \env{article} environment
takes one argument, which is a bibliographical label, and contains commands
describing the article:
  \begin{cmds}
  \item[arttitle] is the title of the included article.
	\item[artauthor] is the author of the included article.
	\item[artyear] is the year the article has been published.
	\item[artpublish] contains information about how or where the article has
	been published.
  \item[artpublishmore] contains extra information about how and where the
  article has been published. The content of this macro is not automatically
  italicized in the list of included articles.
	\item[artcopyright] outputs information about the owner of the copyright of
	the article, after the mention ``Reprinted with kind permission of ''
	%\item[artcopyright] is the copyright notice for the article (the
	%$\copyright$ symbol is automatically added in front of it).
	\item[artpages] is the number of pages of the included articles (this allows
	correct page numbering for content located after the included articles, and
	correct count of the total number of pages)
	\item[arthide] adds the article in the ``list of included articles'', but
	does not make a titlepage for it.
	\end{cmds}

In addition, the \cmd{artmakebib} command can be used for redefining the
layout of the references in the list of included articles, and the
\cmd{artmaketitle} command for controling the layout of the title page of the
included article.

Note that citing an included article works correctly only if the
\opt{loar} option is used.  Currently, citing an included article is not
supported if \pack{biblatex} is in use.
\item[contribution] Defines the author's contribution regarding the included
articles. This environment may appear anywhere in the document, and its
content will be added just after the list of included articles, under the same
heading. Using the \opt{contribbefore} class option, the text will be added
before the list of included articles.
\end{envs}

\section{Useful Internal Commands}

\cls{jydiss} is based on the \cls{book} class; all the features in \cls{book}
are available in \cls{jydiss}. Here is a list of additional packages
which are loaded by \cls{jydiss}, you do not need to load these in your
document:
\begin{itemize}
\item \pack{makeidx}, if \cls{jydiss} is called with option \opt{index}.
\item \pack{babel} with options \opt{finnish} and \opt{english}.
\item \pack{inputenc} with the input encoding character set specified in the
options of \cls{jydiss}.
\item \pack{textcomp}
\item \pack{fontenc} 
\item \pack{palatino} along with \pack{mathpazo} for mathematical fonts
\item \pack{tocloft}
\item \pack{everyshi}
\item \pack{geometry}
\item \pack{remreset}
\item \pack{caption}
\item \pack{ifthen}
\end{itemize}

The following commands are not part of the official API of \cls{jydiss} but
can be useful in some circumstances:

\begin{cmds}
\item[ifpdf] can be used to enable different behaviors depending on wether the
output of \LaTeX\ is PDF (typically, when using \prog{pdflatex}) or not. Here is a
useful example:
\begin{verbatim}
\ifpdf
  \usepackage[pdftex]{hyperref}
  \hypersetup{colorlinks,citecolor=blue}
\else
  \RequirePackage[hypertex]{hyperref}
\fi
\end{verbatim}
\item[HyMakeUppercase] turns text to uppercase in a way which is compatible
with hyperref when using \prog{pdflatex}. Useful especially as an argument to
\cmd{addcontentsline} or \cmd{addtocontents}.

\item[captionsfinnish] contains the definitions of captions in Finnish
language.
\item[captionsenglish] contains the definitions of captions in English
language.
\item[addto] adds definitions to lists of captions (see above). Here is how to
add \cmd{somecaptionname} to Babel's Finnish translated names:

\begin{verbatim}
\addto\captionsfinnish{
  \def\somecaptionname{Joku nimi}
}
\addto\captionsenglish{
  \def\somecaptionname{Some name}
}
\end{verbatim}
After that, \cmd{somecaptionname} will be defined as ``Joku nimi'' when
Finnish language is selected, and as ``Some name'' when English language is
selected.

\item[almostchapter] creates a chapter heading anywhere in a page (i.e., it
does not make a blank page first).
\item[openanychapter] creates a chapter heading at the top of the next page,
wether it is odd-numbered or not (\cmd{chapter} always puts the heading on an
odd-numbered page).
\end{cmds}

\section{Using the provided Bib\LaTeX{} style}

This package comes with an \textbf{experimental} Bib\LaTeX{} style.
Since it is experimental, there may be bugs and infelicities, even
quite severe ones. Send feedback to Antti-Juhani Kaijanaho

Note that citing an included article is not supported at this time
when using the Bib\LaTeX{} style.

To use this style, put the line
\begin{verbatim}
\usepackage[backend=biber,style=jydiss]{biblatex}
\end{verbatim}
somewhere in your thesis preamble.  Using bibtex in the place of biber
as the backend should work, as well, though the use of biber is
recommended.

If you use sources whose original publication date is decades or
centuries earlier than the date of the edition you use, you can
optionally use the \bibfld{origdate} field to specify the original
publication date.  Normally this field is ignored, but you can specify
\opt{citeorigdate=slash} or \opt{citeorigdate=bracket} to
\pack{biblatex} (when using the \pack{jydiss} style) to indicate that
the original publication year should be included in citations.  Under
the option \opt{citeorigdate=slash}, citations of works with
\bibfld{origdate} look like ``Frege (1892/1948)''; under the option
\opt{citeorigdate=bracket}, such citations look like ``Frege
([1892]~1948)''.  Currently, \bibfld{origdate} is not shown in the
bibliography.


In other respects, using the style should be similar to using any of
the standard Bib\LaTeX{} styles.  See the Bib\LaTeX's own manual for
more information.

\section{Using the provided Bib\TeX{} style}

\textbf{Note that Bib\TeX{} is a legacy system.  Use Bib\LaTeX{} and
  Biber if possible.}

This package comes now with an optional Bib\TeX{} style file.  The
style implements the style guidelines for bibliographies specified by
the University of Jyv�skyl� publication unit.

To use the style, specify \cmd{bibliographystyle\{jydiss\}} somewhere
in your document.  You must also load the \pack{natbib} package, since
the style file is intended to be used with an author--year citation
style.

Any standard Bib\TeX{} database ought to work with the style file, but
there are the following additional features available:
\begin{itemize}
\item The \bibfld{language} field may be used in any Bib\TeX\ record, which
  must be a Babel language name (either \bibstr{finnish} or
  \bibstr{english}).  That particular bibliography entry will then be
  rendered in that language, regardless of the document's overall
  language.\footnote{If you want to use the same language for all
    entries, use the \opt{bibweaklang} documentclass option; then the
    entry language will only affect hyphenation.}
\item The \bibfld{url} field may be used in any Bib\TeX\ record to
  give the URL where the document in question is accessible.  It is
  recommended that \bibfld{accessed} is used in conjunction with this
  field.
\item The \bibfld{doi} field may be used in any Bib\TeX\ record to
  specify the Digital Object Identifier (DOI) of the document in
  question.  Do not include the ``doi:'' prefix, please.  It is a good
  idea to provide a DOI where one is available.
\item The \bibfld{accessed} field may be used in any Bib\TeX\ record
  to specify the date when the document in question was accessed
  (through the URL or DOI given separately) by the thesis author.  The
  format is
  ``\texttt\{\textit{year}\texttt\}\texttt\{\textit{month}\texttt\}\texttt\{\textit{day}\texttt\}''.
  (Note that this field is very nonstandard, and works only with the
  provided style file.)
\end{itemize}



\section{Tips and Tricks}

\subsection{\pack{algorithm} and \pack{hyperref}}
\pack{algorithm} (and all packages based on \pack{float}) interact badly with
\pack{hyperref}, since they both redefine \cmd{caption}, producing a lot of
warnings when used with \prog{pdflatex}. You must therefore load these packages
\emph{after} loading \pack{hyperref}. The following piece of black magic will
prevent the warnings.

\begin{verbatim}
% patch of float's \caption to avoid double anchor setting by \refstepcounter
% of float's \caption and in hyperref's \@caption. 
 \begingroup
 \makeatletter
 \def\x#1\refstepcounter#2\@nil{%
 \endgroup
 \def\caption{#1\H@refstepcounter#2}%  
}%
\expandafter\x\caption\@nil
\end{verbatim}

\subsection{Finnish Hyphenation}

If you write your thesis in Finnish, it is useful to have Finnish hyphenation
enabled. You can check if it is enabled by looking for the words
``hyphenation'' and ''finnish'' in the output of \prog{latex}. If it is
similar to the following example, it is already enabled.

\begin{verbatim}
This is e-TeX, Version 3.14159-2.1 (Web2C 7.4.5)
entering extended mode
(./thesis.tex
LaTeX2e <2001/06/01>
Babel <v3.7h> and hyphenation patterns for american, french,
 german, ngerman, finnish, nohyphenation, loaded.
\end{verbatim}

Otherwise, you can enable it by following these instructions (for the
\prog{tetex} distribution in Linux; if you use another distribution, contact
your administrator):

\noindent Edit \url{/etc/texmf/language.dat}, uncomment the line
\smallskip

\verb"%finnish  fi8hyph.tex"
\smallskip

\noindent by removing the \verb"%", and then run
\smallskip

\verb"fmtutil --all"
\smallskip

\noindent if it is in your distribution (it may be a Debian script, I'm not
sure) or if you dont have \prog{fmtutil} go to the directory containing the
tex format files (\url{/var/lib/texmf/web2c} in Debian, otherwise search for a
file called \url{latex.fmt} and go into the directory containing that file)
and run

\begin{verbatim}
tex -ini -jobname=tex -progname=tex tex.ini
tex -ini -jobname=latex -progname=latex latex.ini
etex -ini -jobname=latex -progname=latex *latex.ini
etex -ini -jobname=etex -progname=etex *etex.ini
etex -ini -jobname=elatex -progname=elatex *elatex.ini
pdftex -ini -jobname=pdftex -progname=pdftex pdftex.ini
pdftex -ini -jobname=pdflatex -progname=pdflatex pdflatex.ini
pdfetex -ini -jobname=pdflatex -progname=pdflatex *pdflatex.ini
pdfetex -ini -jobname=pdfetex -progname=pdfetex *pdfetex.ini
pdfetex -ini -jobname=pdfelatex -progname=pdfelatex *pdfelatex.ini
\end{verbatim}

\section{Bugs}

\begin{list}{}{}
\item When using \pack{hyperref} with the \opt{dvips} option, titles in the list of
figures and list of tables do not fold if they are longer than the width of
the text.

\item When using \pack{hyperref}, citations of included articles are linked to the
list of included articles, not to the title page of the article.

\item When citing multiple references, the citation in the body of the
document will be enclosed in square brackets (normal layout) even if it
contains a citation of an included articles (which should be in bold and
without brackets). It will however be prefixed with ``P'' and be identifiable
as a reference to an included article.
\end{list}
\end{document}
