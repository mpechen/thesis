\documentclass{article}
\usepackage[dvips]{graphicx}
\usepackage[latin1]{inputenc}
\usepackage{amssymb,amsmath,array}

\usepackage[table]{xcolor}

\usepackage{hyperref}
\usepackage{bm}
\usepackage{makecell}
\usepackage{longtable}
\usepackage{multirow} 
\usepackage{float}
\usepackage{wasysym}
\usepackage{listings}
\usepackage{rotating}
\usepackage{romannum}
\usepackage{tikz}
\usepackage{subfigure}
\usepackage{amsmath,amsfonts}
\usepackage{scrpage2}
\usepackage{scrtime}
\usepackage{setspace}
\usepackage{natbib}
\usepackage{enumerate}
\usepackage{graphicx}
\usepackage{verbatim}
\usepackage{url}
\usepackage{color}
\usepackage{titlesec}


\begin{document}

A (univariate) time-series $\mathbf{y}$ is a sequence of indexed values
$$
\mathbf{y} = \langle y (t_1),\ldots ,y (t_T) \rangle = \langle y (t_i) \rangle_{i=1}^T  = \langle y_1,\ldots ,y_T \rangle = \langle y_i \rangle_{i=1}^T,
$$
where $t_i$ is the sampling time of the $i$th observation $y_i = y (t_i)$; $T$ determines the number of points, that is, the length of the time series; and $\langle \cdot \rangle$ denotes an ordered row vector.

Further definitions:
\begin{itemize}
\item Let $b_i = b (t_i)$ be the value of the given, black-box predictor of $y_i$, and denote by $e_i = y_i - b_i$ the $i$th prediction error (THINK: $r$ or $e$; what is the difference if we refer this/these as errors versus as residuals?)
\begin{itemize}
\item State here the conditions that we assume on $\langle \mathbf{e} \rangle$
\end{itemize}
\item Define $c_i = c (t_i)$ be the value of a corrector that tries to decrease the $i$th error $e_i$.
\begin{itemize}
\item State here the conditions that we assume on $\langle \mathbf{c} \rangle$
\end{itemize}

\end{itemize}

\end{document}
