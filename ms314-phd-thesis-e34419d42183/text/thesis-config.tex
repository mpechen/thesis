%\usepackage{hyperref}
%%\usepackage{tabularx}
%\usepackage{amsmath, amssymb}
%%\usepackage{glossaries}
%\usepackage{tikz}
%%\usepackage{array}
%%%\usepackage{standalone}
%%\usepackage{cite}
%%%\usepackage{url}
%%\usepackage{graphicx}
%%
%\newtheorem{theorem}{Theorem}
%\newtheorem{definition}{Definition}
%\input{Preamble/preamble}
\usepackage{hyperref}
%%\usepackage{tabularx}
\usepackage{amsmath, amssymb}
%%\usepackage{glossaries}
\usepackage{tikz}
%%\usepackage{array}
%%%\usepackage{standalone}
%%\usepackage{cite}
\usepackage{url}
\usepackage{graphicx}
\usepackage{listings} 
\usepackage{standalone}

\newtheorem{theorem}{Theorem}[chapter]
\newtheorem{lemma}{Lemma}[chapter]
\newtheorem{remark}{Remark}[chapter]
\newtheorem{corollary}{Corollary}[chapter]
\newtheorem{definition}{Definition}[chapter]
\newtheorem{proposition}{Proposition}[chapter]
%\theoremstyle{definition}
\newtheorem{data}{Data}
\newtheorem{example}{Example}

\newcommand{\mnote}[1]{\!^\textrm{\scriptsize\color{Gray}#1}}
\newcommand{\note}[1]{\textrm{\scriptsize\color{Gray}#1}}
\newcommand{\tnote}[1]{\color{Grayy}#1}

% Custom colors
\definecolor{Gray}{rgb}{0.6,0.6,0.6}
\definecolor{Grayy}{rgb}{0.5,0.5,0.5}
\definecolor{deepblue}{rgb}{0,0,0.5}
\definecolor{deepred}{rgb}{0.6,0,0}
\definecolor{deepgreen}{rgb}{0,0.5,0}

%\lstset{emph={trueIndex,root},emphstyle=\color{BlueViolet}}%\underbar} 
% for special keywords
\lstset{language=[LaTeX]Tex,%C++,
    keywordstyle=\color{blue!50!black},%\bfseries,
    basicstyle=\small\ttfamily,
    %identifierstyle=\color{NavyBlue},
    commentstyle=\color{green!50!black}\ttfamily,
    stringstyle=\rmfamily,
    numbers=none,%left,%
    numberstyle=\scriptsize,%\tiny
    stepnumber=5,
    numbersep=8pt,
    showstringspaces=false,
    breaklines=true,
    frameround=ftff,
    frame=single,
    belowcaptionskip=.75\baselineskip
    %frame=L
} 

%------------------ Python style for highlighting ------------------%
\newcommand\pythonstyle{
\lstset{
		language=Python,
		basicstyle=\footnotesize\ttfamily,
		otherkeywords={self},             % Add keywords here
		keywordstyle=\bfseries\color{deepblue}\bfseries,
		commentstyle=\itshape\color{purple!40!black},
		%emph={MyClass,__init__},          % Custom highlighting
		%emphstyle=\ttb\color{deepred},    % Custom highlighting style
		stringstyle=\color{deepgreen},
		frame=single,                         % Any extra options here
		showstringspaces=false,            % 
		belowcaptionskip=.75\baselineskip
}}
% Python environment
\lstnewenvironment{python}[1][]
{
\pythonstyle
\lstset{#1}
}
{}
% Python for external files
\newcommand\pythonexternal[2][]{{
\pythonstyle
\lstinputlisting[#1]{#2}}}
% Python for inline
\newcommand\pythoninline[1]{{\pythonstyle\lstinline!#1!}}
%------------------ END Python style for highlighting ------------------%

\setlength{\tabcolsep}{9pt}
\newcommand{\comment}[1]{}

\numberwithin{equation}{chapter}
\def\proof{\noindent\textit{\textbf{Proof}. }}
\def\proofend{\hfill$\square$\vskip+0.5em}

%--------------------------------- MY ---------------------------------%
% https://www.sharelatex.com/learn/Environments
\newenvironment{boxed1}[1]
{\begin{center}
        #1\\[1ex]
        \begin{tabular}{|p{0.9\textwidth}|}
            \hline\\
        }
        { 
            \\\\\hline
        \end{tabular} 
    \end{center}
}
% \begin{boxed}{Title of the Box}
% This is the text formatted by the boxed environment
% \end{boxed}

%---------------------- Gaussian Distribution ----------------------%
\def \GaussDist {\mathcal{N}(x | \mu, \sigma)  =\frac{1}{(2 \pi \sigma^2)^{1/2}} \exp \Big \{- \frac{(x-\mu)^2}{2 \sigma^2} \Big \}}

%% Right side of the distribution
%\def \GaussDistP {\frac{1}{(2 \pi \sigma^2)^{1/2}} \exp \Big \{- \frac{(x-\mu)^2}{2 \sigma^2} \Big \}}
%\def \GaussDistP {\frac{1}{\sigma \sqrt{2 \pi}} \exp \Big \{- \frac{(x-\mu)^2}{2 \sigma^2} \Big \}}
\def \GaussDistP {\frac{1}{\sigma \sqrt{2 \pi}} e^{ -\frac{(x-\mu)^2}{2 \sigma^2}}}
%Usage:  $p(x|\mu,\sigma) = \GaussDistP$
%---------------------- END Gaussian Distribution ----------------------%

\def \online {on-line }
\def \Online {On-line }

\def \changepoint {change point }
\def \Changepoint {Change point }

\def \cd {changedetection }

%% Make a vector notation
\newcommand {\mvec}[1] {\pmb{#1}}
%\newcommand {\mvec}[1] {\mathbf{#1}}

\newcommand {\lr}[1] {\langle #1 \rangle}
%------------------------------- END MY -------------------------------%
